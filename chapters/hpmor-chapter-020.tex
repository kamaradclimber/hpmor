\chapter{Le Théorème de Bayes}

iftoggle{embeddepigraph}{Ce qui peut-être détruit par Rowling, mérite de l'être.

}{}

\lettrine{H}{arry} leva les yeux vers le plafond gris de la petite pièce, allongé sur le lit portatif et pourtant confortable qui avait été placé ici. Il avait mangé une bonne partie des en-cas du professeur Quirrell -- des friandises faites d'un assemblage complexe de chocolat et d'autres substance, saupoudrées de pincées étincelantes et ornées de petites gemmes en sucre à l'air extrêmement coûteuses et se révélant être, de fait, assez succulentes. Et Harry ne s'était pas senti coupable pour autant~; \emph{ça}, il l'avait \emph{mérité}.

Il n'avait pas essayé de dormir. Il avait dans l'idée qu'il n'aimerait pas ce qu'il verrait s'il fermait les yeux.

Il n'avait pas essayé de lire. Il n'aurait pas été capable de se concentrer.

Amusante, la façon qu'avait le cerveau de Harry de juste continuer à tourner et tourner, ne s'éteignant jamais, quel que soit son état de fatigue. Il devenait plus stupide, mais il refusait de \emph{s'éteindre}.

Mais il y avait une véritable, une réelle sensation de triomphe.

“+1 point au programme Anti-Seigneur-des-Ténèbres-Harry” ne \emph{commençait} même pas à l'exprimer. Harry se demanda ce que le Choixpeau aurait dit, s'il avait pu le mettre sur sa tête \emph{maintenant}.

Pas \emph{étonnant} que le professeur Quirrell ait accusé Harry de s'engager sur la voie d'un Seigneur des Ténèbres. Harry avait été trop lent à la détente, il aurait dû immédiatement voir le parallèle…

\emph{Comprenez que le Seigneur des Ténèbres ne gagna pas, ce jour-là. Son but était d'apprendre les arts martiaux, et pourtant il s'en fut sans avoir reçu une seule leçon.}

Harry était entré dans la salle de Potions avec l'intention d'apprendre l'art des potions. Il était parti sans avoir reçu une seule leçon.

Et le professeur Quirrell l'avait perçu, il l'avait compris avec une précision effrayante, et il s'était avancé et il avait repoussé Harry hors de cette voie, la voie qui l'aurait mené à devenir une copie de Vous-Savez-Qui.

Il y eut un coup contre la porte. «~Le cours est terminé~», dit la voix tranquille du professeur Quirrell.

Harry s'approcha de la porte et se sentit soudain nerveux. Puis la tension diminua, tandis qu'il entendait les pas du professeur Quirrell qui s'éloignaient de la porte.

\emph{Qu'est-ce que c'est que ça~? Est-ce ce qui finira par le faire renvoyer~?}

Harry ouvrit la porte et vit que le professeur Quirrell attendait maintenant à plusieurs pas de lui.

\emph{Le professeur Quirrell le ressent-il lui aussi~?}

Ils traversèrent la plate-forme maintenant déserte jusqu'au bureau du professeur Quirrell sur lequel le ce dernier s'appuya~; et comme la dernière fois, Harry s'arrêta à la limite de l'estrade.

«~Donc~», dit le professeur Quirrell. Il y avait quelque chose d'amical chez lui, même si son visage gardait son sérieux habituel. «~De quoi vouliez-vous me parler, M. Potter~?~»

\emph{J'ai un mystérieux côté obscur}. Mais Harry ne pouvait pas juste le lâcher comme ça.

«~Professeur Quirrell, dit Harry, me suis-je maintenant écarté de la voie qui ferait de moi un Seigneur des Ténèbres~?~»

Le professeur Quirrell regarda Harry.

«~M. Potter~», dit-il solennellement, avec seulement un léger sourire, «~un petit conseil. Une interprétation peut être trop parfaite. Les vraies gens qui viennent de se faire battre et de se faire humilier pendant quinze minutes ne se relèvent pas en pardonnant leurs ennemis avec grâce. C'est le genre de chose qu'on fait quand on essaie de \emph{convaincre} tout le monde que l'on est pas Ténébreux, pas…

--- \emph{Je n'y crois pas~! Vous ne pouvez pas décider que n'importe quelle observation confirmera votre théorie~!}

--- Et c'était un \emph{chouia} trop d'indignation.

--- \emph{Mais qu'est-ce que je dois faire pour vous convaincre~?}

--- Pour me convaincre que vous n'entretenez aucune ambition de devenir un Seigneur des Ténèbres~?~» dit le professeur Quirrell, l'air maintenant franchement amusé. «~J'imagine que vous pourriez lever votre main droite.

--- Quoi~? dit Harry avec un air ébahi. Mais je peux lever ma main droite et jurer que je compte devenir…~» Harry s'interrompit, se sentant plutôt stupide.

«~En effet, dit le professeur Quirrell. Vous pouvez le faire tout aussi facilement, dans un cas comme dans l'autre. Il n'y a rien que vous puissiez faire qui me convaincrait, car je saurais que c'est exactement ce que vous seriez en train d'essayer de faire. Et pour être encore plus précis, bien que je suppose qu'il soit tout juste possible que des gens parfaitement bons existent, même si je n'en ai jamais rencontré, il est tout de même \emph{improbable} que quelqu'un se fasse battre pendant quinze minutes puis qu'il se relève et ressente une brusque montée de clémence envers ses agresseurs. D'un autre côté, il est \emph{moins} improbable qu'un jeune enfant s'imagine que ce serait là le \emph{rôle à jouer} afin de convaincre son enseignant et ses camarades qu'il n'est pas le prochain Seigneur des Ténèbres. Ce qui compte dans un numéro de comédie n'est pas ce à quoi \emph{il ressemble en surface}, M. Potter, mais l'état d'esprit qui rend ce numéro de comédie plus ou moins probable.~»

Harry cligna des yeux. Il venait de se faire expliquer la dichotomie entre l'heuristique de représentativité et la définition Bayésienne de preuve par un sorcier.

«~Mais après tout, dit le professeur Quirrell, n'importe qui peut avoir envie d'impressionner ses amis. Pas besoin que ce soit Ténébreux. Alors, sans que cela constitue un quelconque aveu, M. Potter, dites-moi honnêtement. Quelle pensée vous traversait l'esprit au moment où vous avez déclaré renoncer à toute vengeance~? Cette pensée était-elle une véritable impulsion de pardonner~? Ou votre acte était-il dû à votre conscience de la perception que vos camarades auraient de ce numéro de comédie~?~»

\emph{Il arrive que l'on crée nos propres chants de phénix.}

Mais Harry ne le dit pas à voix haute. Il était clair que le professeur Quirrell ne le croirait pas, et le respecterait probablement moins pour avoir tenté de proférer un mensonge si transparent.

Après quelques moments de silence, le professeur Quirrell sourit avec satisfaction.

«~Croyez-le ou non, M. Potter, dit le professeur, vous n'avez pas besoin de craindre que je découvre votre secret. Je ne vais \emph{pas} vous dire d'arrêter d'essayer de devenir le prochain Seigneur des Ténèbres. Si je pouvais inverser le cours du temps et, d'une façon ou d'une autre, enlever cette ambition de l'esprit de mon moi enfant, le moi de ce présent ne bénéficierait pas de l'altération. Car, aussi longtemps que j'ai pensé que c'était là mon but, il m'a poussé à étudier, à apprendre, à m'affiner et à devenir plus fort. Nous devenons ce que nous sommes censés devenir en suivant nos désirs là où ils nous mènent. C'est ce que Salazar avait compris. Demandez-moi de vous montrer la section de la bibliothèque qui contient les mêmes livres que ceux que je lisais à treize ans, et je vous montrerai le chemin avec joie.

--- Pour l'amour des foutaises~», dit Harry, et il s'assit sur le dur sol de marbre et s'allongea au sol, regardant les lointaines voûtes du plafond. Il aurait voulu s'écrouler de désespoir, mais c'était le mieux qu'il pouvait faire sans se faire mal.

«~Encore trop d'indignation~», nota le professeur Quirrell. Harry ne regardait pas, mais il pouvait entendre le rire réprimé dans sa voix.

Puis Harry comprit.

«~En fait, je crois que je sais ce qui vous embrouille, dit Harry. C'est à vrai dire ce dont je voulais vous parler. Professeur Quirrell, je pense que ce que vous voyez, c'est mon mystérieux côté obscur.~»

Il y eut une pause.

«~Votre… côté… obscur…~»

Harry se redressa. Le professeur Quirrell le regardait avec l'une des expressions les plus étranges que Harry ait jamais vues, surtout sur le visage de quelqu'un d'aussi digne que le professeur.

«~Ça arrive quand je me mets en colère, expliqua Harry. Mon sang devient froid, tout devient froid, tout a l'air parfaitement clair… rétrospectivement, j'ai ça depuis un moment -- la première année où j'ai été à l'école moldue, quelqu'un a essayé de me prendre mon ballon pendant la récréation, et je l'ai tenu dans mon dos et je l'ai frappé dans le plexus solaire, parce que j'avais lu que c'était un point faible, et les autres enfants ne m'ont plus embêté après ça. Et j'ai mordu le professeur de mathématiques quand elle a refusé d'accepter ma domination. Mais ce n'est que récemment que j'ai été assez stressé pour remarquer que c'est un véritable, vous savez, un mystérieux côté obscur, et pas seulement un problème de contrôle de la colère, comme le psychologue de l'école avait dit. Et je n'ai aucun super pouvoir magique quand ça survient, c'est une des premières choses que j'ai vérifiées.~»

Le professeur Quirrell se frotta le nez. «~Laissez-moi y réfléchir~», dit-il.

Harry attendit en silence pendant une minute entière. Il utilisa ce temps pour se lever, ce qui fut plus difficile que ce à quoi il s'était attendu.

«~Eh bien~», dit le professeur Quirrell après un moment. «~J'ai l'impression qu'il y \emph{avait} bien quelque chose que vous pouviez dire qui me convaincrait.

--- J'\emph{ai} déjà deviné que mon côté obscur n'est en fait qu'une partie de moi et que la réponse n'est pas de ne jamais me mettre en colère mais, en l'acceptant, d'apprendre à garder le contrôle de moi-même, je ne suis pas stupide et j'ai déjà vu cette histoire assez de fois pour savoir où elle mène, mais c'est difficile et vous semblez être la personne la plus à même de m'aider.

--- Eh bien… oui… très perspicace de votre part, M. Potter, je dois dire… cet aspect de vous est, comme vous semblez en avoir déjà fait la conjecture, votre intention de tuer, qui comme vous l'avez dit fait partie intégrante de vous…

--- et doit être entraîné~», dit Harry, complétant le motif.

«~Et doit être entraîné, oui.~» Cette étrange expression était toujours sur le visage du professeur Quirrell. «~M. Potter, si vous ne souhaitez réellement pas devenir le prochain Seigneur des Ténèbres, alors quelle était l'ambition pour laquelle vous avez été réparti à Serpentard~?

--- J'ai été réparti à \emph{Serdaigle}~!

--- M. Potter,~» dit le professeur Quirrell, souriant à présent de son sourire caustique habituel, «~je sais que vous avez l'habitude d'être entouré d'idiots, mais s'il vous plaît, ne me prenez pas pour l'un d'eux. La probabilité que le Choixpeau fasse sa première farce en huit cents ans le jour où il se retrouve sur votre tête est tellement faible qu'elle ne mérite pas d'être considérée. J'imagine qu'il est plausible que vous ayez claqué vos doigts et inventé un moyen simple et intelligent de faire échouer les sorts anti-falsification du Choixpeau, bien que je ne puisse moi-même pas imaginer comment vous auriez fait. Mais l'explication de loin la plus probable est que Dumbledore a décidé qu'il n'était pas satisfait de la décision du Choixpeau concernant le Survivant. Puisque c'est évident pour n'importe qui doté du plus petit iota de sens commun, votre secret est en sécurité à Poudlard.~»

Harry ouvrit la bouche, puis la referma, se sentant complètement impuissant. Le professeur Quirrell avait tort, mais tort d'une façon tellement convaincante que Harry commençait à se dire que c'\emph{était} simplement le jugement rationnel à avoir étant donné les informations dont disposait le professeur. Il y avait des occasions, jamais des occasions \emph{prévisibles}, mais ça arrivait quand même, où vous auriez des informations peu probables, et la meilleure conjecture possible serait alors fausse. Si vous aviez un test médical qui ne se trompait qu'une fois sur mille, il se tromperait quand même de temps en temps.

«~Puis-je vous demander de ne jamais répéter ce que je vais vous dire~? dit Harry.

---Absolument, dit le professeur Quirrell. Considérez que vous me l'avez demandé.~»

Harry n'était pas un idiot non plus.

«~Puis-je considérer que vous avez dit oui~?

--- Très bien, M. Potter. Vous pouvez tout à fait le considérer.

--- \emph{Professeur Quirrell}…

--- Je ne répéterai pas ce que vous allez dire~», dit le professeur Quirrell en souriant.

Ils rirent tous deux, puis Harry redevint sérieux.

«~Le Choixpeau semblait penser que j'allais finir Seigneur des Ténèbres à moins que j'aille à Poufsouffle, dit Harry. Mais je ne \emph{veux} pas que cela se produise.

--- M. Potter… dit le professeur Quirrell. Ne le prenez pas mal. Je vous promets que vous ne serez pas noté en fonction de ce que vous allez dire. Je veux seulement connaître votre réponse honnête. Pourquoi pas~?~»

Harry eut à nouveau ce sentiment d'\emph{impuissance}. \emph{Tu ne deviendras point un Seigneur des Ténèbres} était un théorème tellement évident dans son système moral que c'était difficile de décrire les étapes concrètes aboutissant à cette conclusion.

«~Euh, des gens souffriraient~?

--- Vous avez certainement voulu faire souffrir des gens,~» dit le professeur Quirrell. «~Vous vouliez faire souffrir ces petits durs aujourd'hui. Être un Seigneur des Ténèbres signifie que les gens que vous \emph{voulez} voir souffrir souffrent.~»

Harry bredouilla, à la recherche de mots, puis il décida de commencer par l'évidence.

«~Tout d'abord, ce n'est pas parce que je veux faire du mal à quelqu'un que c'est juste de…

--- Qu'est-ce qui rend une action juste, sinon votre désir de le faire~?

--- Ah, dit Harry, l'utilitarisme des préférences.

--- Pardon~? dit le professeur Quirrell.

--- C'est la théorie éthique selon laquelle le bien est ce qui satisfait les préférences de la majorité…

--- Non~», dit le professeur Quirrell. Ses doigts frottaient l'arête de son nez. «~Je ne pense pas que ce soit tout à fait ce que je voulais dire. M. Potter, en définitive, tout le monde agit selon ses désirs. Parfois les gens donnent des noms, comme “juste”, à ce qu'ils font, mais comment serait-il possible que nous soyons mus par \emph{autre chose} que nos désirs~?

--- Ah, évidemment, dit Harry. Je ne pourrais pas \emph{agir} en fonction de considérations morales si elles n'étaient pas capable de m'influencer. Mais cela ne veut pas dire que mon désir de faire du mal à ces Serpentard m'influence \emph{plus} que mes considérations morales~!~»

Le professeur Quirrell cligna des yeux.

«~Sans parler du fait, dit Harry, que devenir un Seigneur des Ténèbres veut dire que de nombreux passants innocents souffriraient eux aussi~!

--- En quoi cela vous importe-t-il~? dit le professeur Quirrell. Qu'ont-ils fait pour vous~?~»

Harry s'esclaffa.

«~Oh, alors \emph{ça}, c'était à peu près aussi subtil que \emph{La Révolte d'Atlas}.

--- Pardon~? dit à nouveau le professeur Quirrell.

--- C'est un livre que mes parents ne me laissaient pas lire parce qu'ils avaient peur qu'il me corrompe, alors bien sûr je l'ai lu, et j'ai été offensé de découvrir qu'ils pensaient que j'allais tomber dans des pièges aussi évident. Bla bla bla, appel à mon sentiment de supériorité, les autres essaient tous de m'opprimer, bla bla bla.

--- Donc vous dites que je dois rendre mes pièges moins évidents~?~» dit le professeur Quirrell. Il tapota du doigt sur sa joue, l'air pensif. «~Je peux y travailler.~»

Ils rirent de concert.

«~Mais pour en rester à la question qui nous préoccupe, dit le professeur Quirrell, \emph{qu'ont-ils} fait pour vous, tous ces gens~?

--- Les gens ont fait des \emph{tonnes} de choses pour moi~! dit Harry. Mes parents m'ont adopté quand mes parents sont morts parce que c'étaient des \emph{gens bien}, et ce serait les trahir que de devenir un Seigneur des Ténèbres~!~»

Le professeur Quirrell resta silencieux pendant un moment.

«~J'avoue que, dit doucement le professeur Quirrell, cette pensée ne me serait jamais venue à votre âge.

--- Je suis navré, dit Harry.

--- Ne le soyez pas, dit le professeur Quirrell. C'était il y a longtemps, et j'ai résolu mes problèmes familiaux à ma façon. Donc c'est l'idée de la désapprobation de vos parents qui vous retient~? Cela veut-il dire que s'ils mouraient dans un accident, il n'y aurait plus rien pour vous empêcher de…

--- Non, dit Harry. Absolument pas. C'est leur \emph{élan de bonté} qui m'a protégé. Cet élan n'est pas présent que chez mes parents. Et c'est cet élan qui serait trahi.

--- En tout cas, M. Potter, vous n'avez pas répondu à ma question initiale, dit enfin le professeur Quirrell. Quelle \emph{est} votre ambition~?

--- Oh, dit Harry. Euh…~» il organisa ses pensées. «~Comprendre tout ce qu'il y a à savoir d'important sur l'univers, appliquer ce savoir pour devenir omnipotent, et utiliser ce pouvoir pour réécrire la réalité, parce que j'ai quelques objections quant à la façon dont elle fonctionne.~»

Il y eut une courte pause.

«~Excusez-moi si c'est une question stupide, M. Potter, dit le professeur Quirrell, mais êtes-vous \emph{sûr} que vous ne venez pas de confesser vouloir devenir le prochain Seigneur des Ténèbres~?

--- C'est seulement si on utilise son pouvoir pour faire le mal, expliqua Harry. Si on l'utilise pour le bien, on est un Seigneur de la Lumière.

--- Je vois~», dit le professeur Quirrell. Il tapota son autre joue. «~Je pense que je peux me contenter de ça. Mais M. Potter, bien que la portée de votre ambition soit digne de Salazar lui-même, comment exactement comptez-vous vous y prendre~? La première étape est-elle de devenir un grand sorcier de combat, ou Chef des Langues-de-plomb, ou Ministre de la Magie, ou…

--- La première étape est de devenir un scientifique.~»

Le professeur Quirrell regardait Harry comme si ce dernier venait de se transformer en chat.

«~Un scientifique~», dit le professeur Quirrell après un moment.

Harry acquiesça.

«~Un \emph{scientifique}~? répéta le professeur Quirrell.

--- Oui, dit Harry. J'accomplirai mes objectifs par le pouvoir… de la \emph{Science}~!

--- Un \emph{scientifique~!}~» dit le professeur Quirrell. Son visage exprimait une authentique indignation, et sa voix était devenue plus forte et plus cassante. «~Vous pourriez être le meilleur de tous mes élèves~! Le plus grand sorcier de combat à sortir de Poudlard de ces cinquante dernières années~! Je ne peux pas vous imaginer gâchant votre vie en blouse blanche, à faire des choses inutiles à des rats~!

--- Hé~! dit Harry. La science ne se résume pas à ça~! Non pas que ce soit \emph{mal} de faire des expériences sur des rats, bien sûr. Mais c'\emph{est} par la science qu'on peut comprendre et contrôler l'univers.

--- Pauvre idiot~», dit le professeur Quirrell d'une voix à l'intensité tranquille et amère. «~Vous êtes un pauvre idiot, Harry Potter.~» Il se passa une main sur le visage, et lorsque sa main fut passée, son visage était plus calme. «~Ou, plus probablement, vous n'avez pas encore découvert votre véritable ambition. Puis-je fortement suggérer que vous essayiez plutôt de devenir un Seigneur des Ténèbres~? Je ferai tout ce qui est en mon pouvoir pour vous aider, dans un esprit d'utilité publique.

--- Vous n'aimez pas la science, dit lentement Harry. Pourquoi~?

--- Ces idiots de Moldus nous tueront tous un jour~!~» la voix du professeur Quirrell était devenue plus puissante. «~Ils vont en finir~! En finir avec tout~!~»

Là, Harry se sentait un peu perdu.

«~De quoi parlons-nous ici, des armes nucléaires~?

--- \emph{Oui}, les armes nucléaires~! le professeur Quirrell criait presque à présent. Même Celui-Dont-On-Ne-Doit-Pas-Prononcer-Le-Nom ne les a jamais utilisées, peut-être parce qu'il ne voulait pas régner sur un tas de cendres~! Elles n'auraient jamais dû être créées~! Et ça ne va faire qu'empirer~!~» Le professeur Quirrell se tenait droit au lieu de s'appuyer sur son bureau. «~Il y a des portes qu'on n'ouvre pas, des sceaux que l'on ne brise pas~! Les idiots qui ne peuvent pas s'empêcher de fouiner sont tués par de moindres périls bien plus tôt, et les survivants savent alors tous qu'il y a des secrets qu'\emph{on ne partage pas} avec quiconque ne possédant pas l'intelligence et la discipline nécessaire pour les découvrir eux-mêmes~! Tout sorcier puissant sait ça~! Même le pire des Seigneurs des Ténèbres sait ça~! Et ces idiots de Moldus ont l'air incapables de comprendre ça~! Les petits idiots impatients qui ont découvert le secret des armes nucléaires ne l'ont pas gardé pour eux-mêmes, ils l'ont révélé à leurs \emph{idiots} de politiciens, et maintenant \emph{nous} devons vivre sous la menace permanente d'une annihilation~!~»

C'était une façon assez différente de voir les choses que celle avec laquelle il avait grandi. Il ne lui était jamais venu à l'esprit que les physiciens nucléaires auraient dû former une conspiration du silence pour garder le secret des armes nucléaires à l'abri de toute personne pas assez intelligente pour être elle-même un physicien nucléaire. La pensée était intrigante, rien de plus. Auraient-ils des mots de passe secrets~? Devraient-ils porter des masques~?

(À vrai dire, pour autant qu'en savait Harry, il y \emph{avait} toutes sortes de secrets incroyablement destructeurs que les physiciens gardaient pour eux-mêmes, et le secret des armes nucléaires était le seul à s'être échappé dans la nature. Aux yeux de Harry, le monde aurait été le même dans un cas comme dans l'autre.)

«~Il faudra que j'y réfléchisse, dit Harry au professeur Quirrell. Cette idée est nouvelle pour moi. Et l'un des secrets \emph{cachés} de la science, transmise par quelques rares professeurs à leurs thésards, est la méthode permettant de ne pas évacuer les idées qu'on n'aime pas à l'instant où on les entend.~»

Le professeur Quirrell cligna à nouveau des yeux.

«~Y a-t-il un type de science que vous \emph{approuvez}~? dit Harry. La médecine, peut-être~?

--- Le voyage spatial, dit le professeur Quirrell. Mais les Moldus semblent traîner les pieds sur le seul projet qui aurait pu laisser les sorciers s'échapper de cette planète avant que les Moldus ne la fassent exploser.~»

Harry hocha la tête. «~Je suis moi aussi un grand fan du programme spatial. Au moins, nous avons ça en commun.~»

Le professeur Quirrell regarda Harry. Quelque chose dans ses yeux étincela.

«~J'aurai votre parole, votre promesse et votre serment de ne jamais parler de ce qui va suivre.

--- Vous l'avez, dit immédiatement Harry.

--- Assurez-vous de respecter votre serment ou vous n'aimerez pas les conséquences dit le professeur Quirrell. Je vais maintenant jeter un sort rare et puissant, non pas sur vous mais sur la salle de classe. Tenez-vous immobile afin de ne pas toucher les limites du sort une fois qu'il aura été jeté. Vous ne devez pas interagir avec la magie que je maintiens. Ne faites que regarder. Sinon, le sort sera interrompu.~» Le professeur Quirrell marqua une pause. «~Et essayez de ne pas tomber.~»

Harry hocha la tête, confus et plein d'anticipation.

Le professeur Quirrell leva sa baguette et dit quelque chose que ni les oreilles ni l'esprit de Harry ne purent saisir, des mots qui traversaient la conscience et disparaissaient dans l'oubli.

Dans un court rayon autour de Harry, le marbre resta semblable à lui-même. Tout le reste de celui-ci disparut, les murs disparurent, le plafond disparut.

Harry se tenait sur un petit cercle de marbre blanc au milieu d'un champ infini d'étoiles qui brûlaient avec une force terrible, inébranlable. Il n'y avait pas de Terre, pas de Lune, pas de Soleil que Harry puisse reconnaître. Le professeur Quirrell se tenait au même endroit qu'avant, flottant au milieu du champ d'étoiles. La Voie Lactée était déjà visible sous la forme d'une grande trace de lumière, et elle devint de plus en plus lumineuse à mesure que la vue de Harry s'ajustait à l'obscurité.

Cette vision serra le cœur de Harry au-delà de tout ce qu'il avait jamais vu.

«~Sommes-nous… dans l'espace…~?

--- Non~», dit le professeur Quirrell. Sa voix était triste et empreinte de révérence. «~Mais c'est une vraie image.~»

Des larmes affluèrent aux yeux de Harry. Il les essuya avec frénésie, il ne voulait pas rater ça à cause de stupides gouttes d'eau brouillant sa vue.

Les étoiles n'étaient plus de petits joyaux disposés sur un immense dôme de velours, comme elles l'étaient dans le ciel nocturne terrestre. Il n'y avait pas de ciel, pas de sphère englobante. Seuls des points de lumière parfaite sur une noirceur parfaite, un vide infini et d'innombrables petits trous qui répandaient la lueur d'inimaginables royaumes lointains.

Dans l'espace, les étoiles \emph{avaient l'air} affreusement, affreusement lointaines.

Harry continua d'essuyer ses yeux, encore et encore.

«~Parfois~», dit le professeur Quirrell d'une voix si douce qu'elle n'était presque pas là, «~quand ce monde vicié me semble inhabituellement empli de haine, je me demande s'il y aurait un autre endroit, loin, où j'aurais dû vivre. Je n'arrive pas à imaginer de quel genre d'endroit il s'agirait, et si je n'arrive même pas à l'imaginer, alors comment puis-je croire qu'il existe~? Et pourtant l'univers est tellement, tellement grand, alors peut-être que cet endroit existe quand même~? Mais les étoiles sont tellement, tellement lointaines. Il faudrait longtemps, très longtemps pour se rendre là-bas, même si je connaissais le chemin. Et je me demande ce que seraient mes rêves si je rêvais pendant longtemps, très longtemps…~»

Bien que cela lui sembla être un sacrilège, Harry parvint à murmurer~: «~Laissez-moi rester ici un peu plus longtemps.~»

Le professeur Quirrell hocha la tête, debout, flottant au milieu des étoiles.

Il était facile d'oublier le petit cercle de marbre où on se tenait, et son propre corps, et de devenir un point de conscience qui aurait pu être immobile ou être mouvant. Quand toutes les distances étaient incalculables, on ne pouvait pas faire la différence.

Il y eut un temps hors du temps.

Puis les étoiles disparurent, et la salle de classe revint.

«~Je suis navré, dit le professeur Quirrell, mais nous sommes sur le point d'avoir de la visite.

--- C'est bon, murmura Harry. Ça m'a suffi.~» Il n'oublierait jamais ce jour, et pas à cause des choses sans importances qui avaient eu lieu plus tôt. Il apprendrait à jeter ce sort même si c'était la dernière chose qu'il pourrait jamais apprendre.

Puis les lourdes portes en chêne de la salle s'arrachèrent de leurs gonds et ricochèrent sur le sol de marbre dans un crissement aigu.

«~\shout{Quirinus~! Comment osez-vous~!}~»

Tel un vaste coup de tonnerre, un ancien et puissant sorcier surgit dans la pièce, son visage reflétant une rage d'une telle incandescence que le regard sévère qu'il avait jeté à Harry plus tôt semblait n'être que peu de chose en comparaison.

Il y eut un déchirement de désorientation dans l'esprit de Harry tandis que la partie qui voulait s'enfuir en hurlant loin de la chose la plus effrayante qu'il ait jamais vue s'en allait, remplacée par une partie de lui capable d'absorber le choc.

\emph{Aucune} des facettes de Harry n'était contente de se faire interrompre en pleine observation stellaire. «~Professeur Albus Percival…~» commença Harry d'un ton de glace.

\emph{BLAM}. La main du professeur Quirrell s'abattit avec force sur son bureau. «~\emph{M. Potter~!} aboya le professeur Quirrell. C'est le \emph{Directeur de Poudlard}, et vous êtes un simple étudiant~! Vous vous adresserez à lui de façon convenable~!~»

Harry regarda le professeur Quirrell.

Le professeur Quirrell regardait Harry avec sévérité.

Aucun d'eux ne souriait.

Les longues enjambées de Dumbledore s'étaient arrêtées là où Harry se tenait face à l'estrade, et le professeur Quirrell se tenait à côté de son bureau. Le Dumbledore les regarda tous deux avec un air choqué.

Lentement, l'expression de Dumbledore passa d'un air capable de vaporiser l'acier à un air simplement colérique. «~J'ai entendu des étudiants dire que cet homme avait poussé des Serpentard à t'infliger des sévices~! Qu'il t'avait interdit de te défendre~!~»

Harry hocha la tête.

«~Il savait exactement ce qui clochait chez moi et il m'a montré comment le régler.

--- Harry, \emph{de quoi parles-tu}~?

--- Je lui apprenais à perdre, dit sèchement le professeur Quirrell. C'est une aptitude des plus importantes.~»

Il était clair que Dumbledore ne comprenait toujours pas, mais sa voix était descendue d'un registre. «~Harry… dit-il lentement. Si le professeur de Défense a proféré la moindre menace t'empêchant de te plaindre…~»

\emph{Espèce de lunatique, après une journée comme celle-ci pensez-vous vraiment que…}

«~Professeur~», dit Harry, essayant d'avoir l'air décontenancé, «~ce qui cloche chez moi n'est pas que je garde le silence face à des professeurs abusifs.~»

Le professeur Quirrell gloussa. «~Pas parfait, M. Potter, mais pas mal pour votre premier jour. Professeur, êtes-vous resté assez longtemps pour entendre parler des cinquante-et-un points décernés à Serdaigle, où êtes-vous sorti en trombe dès que vous avez entendu la première partie~?~»

Un air déconcerté passa sur le visage de Dumbledore, suivi par de la surprise. «~Cinquante-et-un points pour Serdaigle~?~»

Le professeur Quirrell hocha la tête. «~Il ne s'y attendait pas, mais cela me semblait approprié. Dites au professeur McGonagall que l'histoire de ce que M. Potter a traversé pour récupérer les points perdus permettra aussi bien d'accomplir le but qu'elle avait en tête. Non, professeur, M. Potter ne m'a rien dit. Il est facile de discerner quelle partie des événements d'aujourd'hui sont le fruit de son œuvre, tout comme je sais que vous avez suggéré le compromis final. Mais je me demande comment M. Potter a bien pu parvenir à avoir la main haute sur vous et sur Rogue, et comment le professeur McGonagall a ensuite réussi à avoir la main haute sur lui.~»

Sans qu'il sache comment, Harry parvint à contrôler son expression. Était-ce \emph{si} évident que ça pour un véritable Serpentard~?

Dumbledore s'approcha de Harry, le regard scrutateur.

«~Tu sembles un peu pâle, Harry,~» dit le vieux sorcier. Il regarda fixement Harry. «~Qu'as-tu eu au déjeuner aujourd'hui~?

--- Quoi~?~» dit Harry, son esprit chancelant sous l'effet de la confusion soudaine. Pourquoi Dumbledore l'interrogerait-il au sujet d'agneau frit et de brocoli en tranches fines quand c'était à peu près la \emph{dernière} raison possible qu'il…

Le vieux sorcier se redressa. «~Aucune importance, alors. Je pense que tu vas bien.~»

Le professeur Quirrell toussa, fortement et délibérément. Harry regarda le professeur et vit qu'il fixait Dumbledore avec un air sévère.

«~\emph{Ah-hem~!}~» dit à nouveau le professeur Quirrell.

Dumbledore et le professeur Quirrell se fixèrent l'un l'autre, et quelque chose sembla passer entre eux.

«~Si vous ne lui dites pas, dit alors le professeur Quirrell, je le ferai, même si vous me renvoyez pour ça.~»

Dumbledore soupira et se retourna vers Harry. «~Je vous demande pardon d'avoir envahi votre intimité mentale, M. Potter, dit le directeur d'un ton formel. Je n'avais d'autre but que de déterminer si le professeur Quirrell avait fait de même.~»

\emph{Quoi~?}

La confusion subsista pendant la durée exacte nécessaire à Harry pour comprendre ce qui venait de se produire.

«~\emph{Vous…}

--- Tout doux, M. Potter~», dit le professeur Quirrell. Mais il regardait Dumbledore d'un air dur.

«~On confond souvent la Legilimancie et le bon sens, dit le directeur. Mais elle laisse des traces qu'un autre Legilimens habile peut détecter. C'est tout ce que je cherchais à savoir, M. Potter, et je vous ai posé une question sans rapport avec le sujet afin de m'assurer que vous ne penseriez à rien d'important pendant que je regardais.

--- \emph{Vous auriez dû me demander d'abord~!}~»

Le professeur Quirrell secoua la tête.

«~Non, M. Potter, le professeur Dumbledore était assez justifié dans son inquiétude, et s'il vous avait demandé la permission, vous auriez pensé exactement à ces choses que vous ne souhaitez pas qu'il voie.~» La voix du professeur Quirrell devint plus coupante. «~Je suis bien plus préoccupé, professeur, par le fait que vous n'ayez pas ressenti le besoin de le lui dire après coup~!

--- Vous avez maintenant rendu plus difficile la tâche d'envahir son intimité mentale en de futures occasions~», dit Dumbledore. Il offrit un regard froid au professeur Quirrell. «~Je me demande si c'était là votre intention.~»

L'expression du professeur Quirrell était implacable. «~Il y a trop de Legilimens dans cette école. J'insiste pour que M. Potter reçoive des instructions en Occlumancie. Me permettrez-vous d'être son tuteur~?

--- Absolument pas, dit immédiatement Dumbledore.

--- C'est ce que je pensais. Alors, puisque \emph{vous} l'avez privé de mes services gratuits, \emph{vous} paierez pour les leçons particulières de M. Potter, données par un instructeur en Occlumancie agréé.

--- De tels services ne sont pas donnés~», dit Dumbledore, regardant le professeur Quirrell avec surprise. «~Bien que je dispose de certaines connexions…~»

Le professeur Quirrell secoua fermement la tête. «~Non. M. Potter demandera à son responsable des comptes à Gringotts de lui recommander un instructeur neutre. Avec tout le respect que je vous dois, professeur Dumbledore, après les événements de ce matin, je dois protester contre la possibilité que vous ou vos amis aient accès à l'esprit de M. Potter. Je dois aussi insister pour que l'instructeur fasse un Serment Inviolable et qu'il accepte de recevoir un sortilège d'Amnésie immédiatement après chaque cours.~»

Dumbledore fronçait les sourcils.

«~De tels services sont \emph{extrêmement} coûteux, comme vous le savez bien, et je ne peux m'empêcher de me demander pourquoi \emph{vous} les jugez nécessaires.

--- Si l'argent est un problème, dit Harry, j'ai quelques idées pour amasser rapidement de grandes quantités d'argent…

--- Merci Quirinus, votre sagesse est maintenant évidente et je suis désolé de l'avoir mise en doute. Et votre préoccupation pour Harry Potter vous fait honneur.

--- De rien, dit le professeur Quirrell. J'espère que vous ne verrez pas d'objection à ce que je continue à le placer au centre de mon attention.~» Le visage du professeur Quirrell était maintenant très sérieux, et très immobile.

Dumbledore baissa les yeux vers Harry.

«~C'est mon souhait à moi aussi, dit Harry.

--- Alors il en sera ainsi…~» dit lentement le vieux sorcier. Quelque chose d'étrange passa sur son visage. «~Harry… tu dois te rendre compte que si tu choisis cet homme pour enseignant et pour ami, pour premier mentor, alors d'une façon ou d'une autre tu le perdras, et la façon dont tu le perdras pourra ou ne pourra pas te permettre de jamais le retrouver.~»

Harry n'y avait pas pensé. Mais il y \emph{avait} cette malédiction sur le poste de Défense… une malédiction qui avait apparemment fonctionné avec une régularité parfaite pendant des décennies…

«~Probablement, dit calmement le professeur Quirrell, mais il m'aura à son entière disposition tant que je tiendrai.~»

Dumbledore soupira. «~J'imagine qu'au moins c'est économique, puisqu'en tant que professeur de Défense vous êtes \emph{déjà} maudit, d'une manière inconnue.~»

Harry dut faire un grand effort pour ne pas trahir sa pensée quand il se rendit compte de ce que Dumbledore avait réellement sous-entendu.

«~J'informerai madame Pince que M. Potter est autorisé à obtenir des livres sur l'Occlumancie, dit Dumbledore.

--- Il y a un entraînement préliminaire que vous devez faire par vous-même, dit le professeur Quirrell à Harry. Et je suggère que vous vous y atteliez rapidement.~»

Harry acquiesça.

«~Je vais vous quitter, alors~», dit Dumbledore. Il hocha la tête en direction de Harry et du professeur Quirrell, et s'en fut, marchant plutôt lentement.

«~Pourriez-vous jeter le sort à nouveau~?~» dit Harry à l'instant où Dumbledore fut parti.

«~Pas aujourd'hui, dit calmement le professeur Quirrell, et pas demain non plus, j'en ai bien peur. Le lancer me demande beaucoup d'énergie mais le maintenir est plus facile, et je préfère généralement donc le maintenir aussi longtemps que possible. Cette fois-ci, je l'ai lancé sous le coup d'une impulsion. Si j'avais réfléchi et que je m'étais rendu compte que nous pourrions être interrompus…~»

Dumbledore était maintenant la personne que Harry aimait le moins du monde entier.

Ils soupirèrent tous les deux.

«~Même si je ne le vois jamais plus, dit Harry, je vous en serai toujours reconnaissant.~»

Le professeur Quirrell hocha la tête.

«~Avez-vous entendu parler du programme Pioneer~? dit Harry. C'étaient des sondes qui survolaient différentes planètes et prenaient des images. Les deux sondes allaient finir sur des trajectoires qui les mèneraient hors du système solaire et dans l'espace interstellaire. Alors ils ont mis une plaque en or sur les sondes, avec l'image d'un homme, et d'une femme, et des instructions pour trouver notre soleil dans la galaxie.~»

Le professeur Quirrell resta silencieux un moment, puis il sourit. «~Dites-moi, M. Potter, pouvez-vous deviner ce qui m'a traversé l'esprit quand j'ai terminé d'assembler les trente-sept choses que je ne ferai jamais une fois devenu un Seigneur des Ténèbres~? Mettez-vous à ma place -- imaginez-vous à ma place -- et devinez.~»

Harry s'imagina regardant une liste de trente-sept choses à ne jamais faire une fois devenu un Seigneur des Ténèbres.

«~Vous avez décidé que si vous deviez respecter \emph{toute} la liste \emph{tout} le temps, ça n'aurait pas vraiment de sens de devenir un Seigneur des Ténèbres, dit Harry.

--- \emph{Précisément}~», dit le professeur Quirrell. Il souriait. «~Alors je vais transgresser la règle numéro deux -- qui était juste “ne te vante pas” -- et vous parler de quelque chose que j'ai fait. Je ne vois pas comment votre connaissance de ce fait pourrait poser problème. Et je soupçonne fortement que vous l'auriez deviné de toute façon après que nous soyons devenus proches. Néanmoins… je veux entendre votre serment de ne jamais parler de ce que je vais vous dire.

--- Vous l'avez~!~» Harry sentait que ça allait être \emph{vraiment} bon.

«~Je suis inscrit à un bulletin d'information moldu qui me tient informé des progrès en matière de voyage spatial. Je n'ai pas entendu parler de Pioneer 10 avant qu'ils ne signalent son lancement. Mais j'ai alors découvert que Pioneer 11 allait aussi quitter le système solaire pour toujours,~» dit le professeur Quirrell, et il avait le plus large sourire que Harry l'avait jamais arboré, «~je me suis introduit à la NASA, eh oui, et j'ai jeté un charmant petit sort sur cette charmante plaque d'or qui la fera durer beaucoup plus longtemps qu'elle n'aurait autrement tenu.~»

…

…

…

«~Oui~», dit le professeur Quirrell, qui semblait maintenant faire quinze mètres de plus, «~je me disais que vous pourriez réagir ainsi.~»

…

…

…

«~M. Potter~?

--- … je ne sais pas quoi dire.

--- “Vous avez gagné” me semblerait de circonstance, dit le professeur Quirrell.

--- Vous avez gagné, dit immédiatement Harry.

--- Vous voyez~? dit le professeur Quirrell. Nous ne pouvons qu'imaginer le pétrin immense dans lequel vous vous seriez retrouvé si vous n'aviez pas été capable de dire ça.~»

Ils rirent tous les deux.

Une autre pensée vint à Harry.

«~Vous n'avez ajouté aucune information supplémentaire à la plaque~?

--- Information supplémentaire~?~» dit le professeur Quirrell, comme si l'idée ne lui était jamais venue avant et qu'il était assez intrigué.

Ce qui rendit Harry plutôt suspicieux, étant donné que ça \emph{lui} avait pris moins d'une minute pour y penser.

«~Peut-être que vous avez inclus un message holographique, comme dans \emph{Star Wars}~? dit Harry. Ou… Hmm. Un portrait a l'air de contenir autant d'information qu'un cerveau humain… vous n'auriez pas pu ajouter un poids supplémentaire à la sonde, mais peut-être avez-vous pu transformer une partie déjà présente en un portrait de vous-même~? Ou vous avez trouvé un volontaire succombant à une maladie mortelle, l'avez introduit à la NASA, et avez jeté un sort pour vous assurer que son \emph{fantôme} finirait dans la plaque…

--- M. Potter~», dit le professeur Quirrell, sa voix soudain coupante, «~un sort nécessitant la mort d'un humain serait certainement considéré comme de la Magie Noire par le Ministère, peu importe les circonstances. On ne devrait pas entendre un étudiant discuter de choses pareilles.~»

Et ce qui était incroyable dans la façon dont le professeur Quirrell avait dit ça, c'était la perfection avec laquelle il maintenait le déni plausible. Il l'avait dit d'un ton qui correspondait exactement à celui de quelqu'un ne désirant pas discuter de tels sujets et pensant que les étudiants devraient s'en tenir à distance. Harry ne savait honnêtement \emph{pas} si le professeur Quirrell attendait juste que Harry ait appris à protéger son esprit avant de lui en parler.

«~J'ai compris, dit Harry. Je ne parlerai de cette idée à personne d'autre.

--- Merci d'être discret en ce qui concerne toute cette affaire, M. Potter, dit le professeur Quirrell. Je préfère vivre sans attirer l'attention publique. Vous ne trouverez rien dans les journaux concernant Quirinus Quirrell qui date d'avant le jour où j'ai décidé qu'il était temps pour moi d'enseigner la Défense à Poudlard.~»

Ça semblait un peu triste, mais Harry comprenait. Puis il comprit ce que cela impliquait~:

«~Et alors exactement combien de trucs géniaux \emph{avez-vous} fait dont personne d'autre n'a jamais entendu…

--- Oh, quelques-uns, dit le professeur Quirrell. Mais je pense que c'est assez pour aujourd'hui, M. Potter, je vous avoue que je me sens un peu fatigué…

--- Je comprends. Et \emph{merci}. Pour \emph{tout}.~»

Le professeur Quirrell hocha la tête, mais il se penchait de plus en plus sur son bureau.

Harry prit rapidement congé.~
%  LocalWords:  arry

\partchapter{Travail de groupe}{I}

iftoggle{embeddepigraph}{J.K. Rowling, si un homme vous importune, vous pouvez penser bleu, compter jusqu'à deux et chercher d'autres cieux

\iftoggle{embedepigraphs}{\newpage}{}
}{}

\lettrine{N}{ous} étions le lundi 3 novembre et bientôt, Harry Potter, Drago Malfoy et Hermione Granger, les trois puissants de cette année, allaient débuter leur combat pour la domination suprême.
(Harry avait été légèrement agacé par la façon dont le Survivant avait été rétrogradé du statut de dominant suprême à celui de membre d'un groupe de trois rivaux simplement à cause de sa participation à un concours, mais il comptait récupérer ce statut très bientôt).

Le champ de bataille était une section de la forêt non-Interdite particulièrement dense, car le professeur Quirrell pensait que pouvoir voir toutes les forces ennemies aurait été trop ennuyeux, même pour une première bataille.

Tous les élèves qui ne faisaient pas à proprement parler \emph{partie} d'une armée de première année étaient installés non loin et regardaient des écrans que le professeur Quirrell avait installés. Mis à part trois Gryffondor en quatrième année qui étaient pour le moment malades et restreints par madame Pomfresh aux lits de l'infirmerie. Mis à part eux, tout le monde était là.

Les élèves combattants étaient apprêtés non pas de leurs robes ordinaires mais d'uniformes de camouflage Moldus que le professeur Quirrell avait dénichés quelque part et qu'il avait fournis en quantité et en variété suffisante pour convenir à tout le monde. Non pas que les élèves se soient inquiétés de déchirures et de tâches, les sortilèges étaient là pour ça. Mais comme le professeur Quirrell l'avait expliqué à ces nés-sorciers surpris, de beaux habits dignes n'étaient pas efficaces lorsqu'il s'agissait de se cacher dans une forêt ou d'éviter des arbres en courant.

Et à la poitrine de chaque uniforme, un écusson portant le nom et l'emblème de l'armée du soldat. Un \emph{petit} écusson. Si vous vouliez que vos soldats portent, disons, des rubans colorés afin qu'ils puissent s'identifier les uns les autres avec certitude, au risque que l'ennemi fasse main basse sur les rubans, ça ne tenait qu'à vous.

Harry avait essayé d'avoir le nom “Armée du Dragon”.

Drago avait fait une crise en disant que cela dérouterait tout le monde.

Le professeur Quirrell avait décrété que Drago avait un droit préalable au nom s'il le souhaitait.

Donc maintenant Harry combattait l'Armée du Dragon.

Ce n'était probablement pas bon signe.

Pour leur emblème, au lieu du trop-évident dragon crachant du feu, Drago avait choisi de s'en tenir au feu. Élégant, euphémique, mortel. \emph{C'est ce qui reste après notre passage}. Très Malfoy.

Harry, après avoir considéré des alternatives telles que le 501\textsuperscript{e} Bataillon Provisionnel et Les Servants Mortels de Harry, avait décidé que son armée serait connue sous l'appellation simple et digne de Légion du Chaos.

Leur emblème était une main dont les doigts étaient prêts à claquer.

Il était \emph{universellement} accepté que ce n'était pas bon signe.

Harry avait ardemment prévenu Hermione que le fait qu'elle soit une fille réputée pour être gentille rendrait probablement les jeunes garçons servant sous son commandement nerveux et qu'elle devrait choisir quelque chose d'effrayant qui les rassurerait quant à sa coriacité et les rendrait fiers de faire partie de son armée, comme le Commando du Sang ou quelque chose comme ça.

Hermione avait appelé son armée le Régiment du Soleil.

Leur emblème était un smiley.

Et dans dix minutes, ils seraient en guerre.

Harry se tenait à l'emplacement de départ qui leur avait été indiqué, une large et lumineuse clairière ouverte où se trouvaient de vieilles souches pourrissantes qui avaient été dégagées dans un but inconnu sur un sol recouvert de petites constellations de feuilles balayées par le vent et des restes gris et secs d'une herbe qui n'avait pas survécu à la chaleur de l'été placée sous un soleil qui brillait puissamment depuis le ciel.

Autour de lui se trouvaient les vingt-trois soldats que le professeur Quirrell lui avait assignés. Presque tout Gryffondor s'était bien sûr inscrit, et plus de la moitié de Serpentard, et moins de la moitié de Poufsouffle, et une poignée de Serdaigle. Dans l'armée de Harry se trouvaient douze Gryffondor, six Serpentard, quatre Poufsouffle et un Serdaigle mis à part lui… non pas qu'il ait été possible de les distinguer en regardant leur uniforme. Pas de rouge, pas de vert, pas de jaune et pas de bleu. Juste des motifs de camouflages Moldus et un écusson sur la poitrine représentant une main prête à claquer des doigts.

Harry toisa ses vingt-trois soldats du regard. Ils portaient tous le même uniforme dénué de signe identificateur mis à part cet unique écusson.

Et Harry sourit alors, car il avait compris le dessein de cette partie du plan directeur du professeur Quirrell~; et Harry allait en tirer un avantage pour \emph{ses propres} desseins.

Un épisode légendaire de psychologie sociale était nommé l'expérience de Robbers Cave. Il avait eu lieu durant l'après-coup abasourdi de la seconde guerre mondiale, dans le but d'enquêter sur les causes et les remèdes des conflits entre groupes. Le scientifique avait organisé une colonie d'été pour vingt-deux garçons venus de vingt-deux écoles différentes, sélectionnant uniquement ceux qui venaient de familles stables de classe moyenne. La première phase de l'expérience avait consisté à découvrir ce qui était nécessaire pour \emph{enclencher} un conflit entre des groupes. Les vingt-deux garçons avaient été divisés en deux groupes de onze…

… et cela avait largement suffi.

Les hostilités avaient commencé au moment où les deux groupes étaient devenus conscients de la présence de l'autre groupe dans le parc et des insultes avaient été jetées dès la première rencontre. Ils s'étaient eux-mêmes nommés les Aigles et les Serpents à Sonnettes (ils n'avaient pas eu besoin de nom lorsqu'ils avaient pensé être seuls dans le parc) et avaient entrepris de développer des stéréotypes de groupes en contraste les uns avec les autres. Les Serpents à Sonnettes se voyaient comme des durs à cuire extrêmement grossiers, les Aigles se voyant par conséquent droits dans leurs bottes et bien élevés.

L'autre partie de l'expérience avait consisté à trouver comment résoudre les conflits de groupe. Rassembler les garçons pour qu'ils regardent des feux d'artifice n'avait pas fonctionné. Ils s'étaient contentés de se hurler les uns sur les autres et étaient restés à l'écart. Ce qui \emph{avait} fonctionné, c'était de leur annoncer qu'il y avait peut-être des vandales dans le parc, et aussi de leur annoncer que les deux groupes devraient travailler ensemble pour résoudre un défaut dans le système d'irrigation du parc. Une tâche commune, un ennemi commun.

Harry soupçonnait fort que le professeur Quirrell avait très bien compris ce principe lorsqu'il avait choisi de créer \emph{trois} armée par année.

\emph{Trois} armées.

Pas \emph{quatre}.

Et certainement \emph{pas} divisées par maisons… sauf qu'à part M. Crabbe et M. Goyle, aucun Serpentard n'avait été assigné à Drago.

C'était ce genre de chose qui rassurait Harry quant à l'idée que le professeur Quirrell, en dépit de ses airs sombres et de sa prétention à la neutralité dans le conflit entre le Bien et le Mal, était secrètement du côté du Bien, non pas que Harry aurait jamais osé le dire à voix haute.

Et Harry décida profiter au maximum du plan du professeur Quirrell et de définir une identité de groupe à \emph{sa} façon.

Les Serpents à Sonnettes, une fois qu'ils avaient rencontré les Aigles, avaient commencé à se voir comme des durs à cuire, et ils s'étaient comportés en accord avec cette notion.

Les Aigles s'étaient considérés comme des gens bien élevés.

Et dans cette lumineuse clairière, éparpillés autour des vieilles souches pourrissantes, détourés par le soleil qui brillait puissamment depuis le ciel, le général Potter et ses vingt-trois soldats étaient disposés selon une configuration qui ne ressemblait même pas de loin à une formation. Certains soldats étaient debout, d'autres assis, certains se tenaient sur une jambe juste pour être différents.

Après tout, c'était la Légion du \emph{Chaos}.

Et s'il n'y avait une bonne \emph{raison} de former des petites lignes bien propres, avait dit Harry avait dédain, il n'y aurait pas de petites lignes bien propres.

Harry avait divisé l'armée en six escouades de quatre soldats, chaque escouade dirigée par un Suggéreur d'escouade. Toutes les troupes avaient l'ordre strict de désobéir à tout ordre reçu si obéir ne leur semblait pas être une bonne idée à ce moment précis, y compris cet ordre… à moins que Harry ou que le Suggéreur d'escouade ne préfixe l'ordre par «~Merlin a dit~», auquel cas on était censé vraiment obéir.

La stratégie principale de la Légion du Chaos était de se diviser et de venir de multiples directions, changeant de vecteurs au hasard et jetant les sorts de sommeil approuvés aussi rapidement qu'il fût possible de recharger sa force magique. Et si vous voyez une chance de distraire ou de semer la confusion chez l'ennemi, vous la preniez.

Rapides. Créatifs. Imprévisibles. Hétérogènes. N'obéissez pas seulement aux ordres, décidez si ce que vous êtes en train de faire \emph{maintenant} a un sens.

Harry prétendait que c'était l'optimum de l'efficacité militaire, mais il n'en était pas aussi certain qu'il le prétendait… enfin, il avait reçu une opportunité en or de changer la façon dont certains élèves \emph{se voyaient eux-mêmes}, et c'était ainsi qu'il comptait l'utiliser.

Cinq minutes avant la guerre, selon la montre de Harry.

Le général Potter marcha (d'un pas de civil) jusqu'à l'endroit où sa force aérienne attendait, tendue, balais déjà fermement tenus en main.

«~Toutes les escadrilles au rapport~», dit le général Potter. Ils avaient répété cela pendant leur session d'entraînement du samedi.

«~Red Leader prêt, dit Seamus Finnigan qui n'avait aucune idée de ce que cela pouvait bien vouloir dire.

--- Red Cinq prêt, dit Dean Thomas qui avait attendu toute sa vie de pouvoir le dire.

--- Green Leader prêt, dit Theodore Nott avec raideur.

--- Green Quarante-et-Un prêt, dit Tracey Davis.

--- Je veux vous voir dans les airs à l'instant où on entendra la cloche, dit le général Potter. N'engagez pas le combat, je répète, n'engagez pas le combat. Fuyez si vous recevez des tirs.~» (Bien sûr, \emph{personne} ne jetterait un sort de sommeil au balai ou au sorcier~; on jetterait un sort qui donnait une lueur rouge temporaire à tout ce qu'il touchait. Si le balai ou le sorcier était touché, il était éliminé de la guerre.) «~Red Leader et Red Cinq, volez vers l'armée de Malfoy aussi vite que possible, restez aussi haut que possible tout en les gardant dans votre champ de vision, revenez à l'instant où vous êtes certains de ce qu'ils sont en train de faire. Green Leader, faites de même pour l'armée de Granger. Green Quarante-et-Un, restez au-dessus de nous et surveillez tout soldat ou balai approchant, vous êtes le seul autorisé à tirer. Et souvenez-vous, je n'ai dit “Merlin a dit” avant aucun de ces ordres, mais on a \emph{vraiment} besoin de ces informations. Pour le Chaos~!

--- Pour le Chaos~!~» répondirent les quatre en écho à des degrés variables d'enthousiasme.

Harry s'était attendu à ce que Hermione lance une attaque immédiate sur Drago, auquel cas il mettrait ses troupes en position et commencerait à la soutenir, mais seulement après qu'elle eut subit des pertes sévères et causé quelque dommage. Si possible, il donnerait à l'opération l'apparence d'un sauvetage héroïque~; après tout, il ne faudrait pas laisser Soleil croire que Chaos n'était pas son ami.

Mais juste au cas où elle ne le ferait \emph{pas}… eh bien c'était pour cela que la Légion du Chaos ne bougerait pas avant le retour de Green Leader.

Les mouvements de Drago seraient dans son intérêt personnel. Il était prévisible qu'il prépare son armée à se défendre contre Hermione~; il pouvait s'être rendu compte, ou ne pas s'être rendu compte, que Harry avait menti lorsqu'il avait parlé d'attendre que la bataille soit terminée avant d'attaquer. Harry avait tout de même mis deux balais sur l'Armée du Dragon juste au cas où ils \emph{feraient} quelque chose, et juste au cas où Drago ou M. Goyle ou M. Crabbe seraient assez bons pour abattre un balai haut dans le ciel.

Mais c'était le général Granger qui était imprévisible, et Harry ne pourrait pas bouger avant de savoir ce qu'elle faisait.

\later

Au cœur de la forêt, entouré d'ombres dansantes projetant des motifs au sol tandis qu'une voûte de feuilles ondulait haut au-dessus de leurs têtes, le général Malfoy se tenait là où les arbres étaient relativement moins denses. Il regardait ses troupes avec une calme satisfaction. Six unités de trois membres chacune, l'unité aérienne composée de quatre membres (à laquelle Grégory était assigné), et l'unité de commandement qui comprenait Vincent et lui-même. Ils n'avaient répété que pendant une courte période le samedi mais Drago avait la certitude d'avoir su expliquer les bases. Restez avec vos équipiers, surveillez leurs arrières et faites-leur confiance pour surveiller les vôtres. Bougez comme un seul homme. Obéissez aux ordres et ne révélez aucune frayeur. Visez, tirez, bougez, visez encore, tirez encore.

Les six unités se placèrent le long d'un périmètre défensif autour de Drago, observant la forêt avec vigilance. Ils se tenaient, dos à dos, les baguettes maintenues basses jusqu'à ce que vienne le moment de frapper.

Ils ressemblaient déjà remarquablement aux unités d'Aurors dont Drago avait observé l'entraînement pendant les inspections de son père.

Chaos et Soleil ne sauraient pas ce qui s'était abattu sur eux.

«~Votre attention, s'il vous plaît~», dit le général Malfoy.

Les six unités se détachèrent et pivotèrent vers Drago~; les visages des membres des forces aériennes se retournèrent, balais déjà en main.

Drago avait décidé d'attendre de gagner la première bataille avant d'exiger des saluts martiaux, lorsque les Gryffondor et des Poufsouffle seraient plus prêts à accepter de saluer un Malfoy.

Mais ses soldats se tenaient déjà si droits, surtout les Gryffondor, que Drago se demanda s'il avait eu besoin de délayer l'ordre. Grégory avait écouté en silence et avait confié à Drago que son initiative consistant à soutenir Harry Potter en cours de Défense la fois où le professeur Quirrell avait enseigné à Harry comment perdre lui avait déjà valu d'être reconnu comme un chef acceptable. Du moins si vous vous retrouviez assigné à son armée. \emph{Tous les Serpentard ne se ressemblent pas~; il y a des Serpentard et des Serpentard} était la phrase que répétaient les Gryffondor de son armée à leurs camarades de maison.

Drago était franchement stupéfait par la facilité de la chose. Il avait d'abord protesté contre le fait de n'avoir aucun Serpentard, et le professeur Quirrell lui avait dit que s'il voulait être le premier Malfoy à obtenir le contrôle total de la politique du pays, il lui faudrait apprendre comment gouverner les trois autres quarts de la population. C'était ce genre de chose qui rassurait Drago quant au fait que le professeur Quirrell avait bien plus de sympathie pour les gentils qu'il ne le laissait paraître.

La bataille en elle-même ne serait pas simple, en particulier si Granger commençait par attaquer les Dragons. Drago s'était torturé l'esprit, hésitant à immédiatement engager toutes ses forces dans une frappe préventive contre Granger, mais il s'était inquiété que 1) Harry l'ait complètement induit en erreur au sujet de ce que Granger risquait de faire et que 2) Harry l'ait complètement induit en erreur en disant qu'il attendrait que Granger attaque avant de participer à la bataille.

Mais l'Armée du Dragon avait une arme secrète, trois à vrai dire, ce qui suffirait peut-être à remporter la victoire même s'ils étaient attaqués par les deux armées en même temps…

Il était presque l'heure, et cela voulait dire que le moment du discours d'avant-bataille qu'il avait composé et mémorisé était venu.

«~La bataille est sur le point de commencer~», dit Drago. Sa voix était calme et précise. «~Souvenez-vous de tout ce que moi-même, M. Crabbe et M. Goyle vous avons montré. Une armée gagne parce qu'elle est disciplinée et mortelle. Le général Potter et la Légion du Chaos ne seront pas disciplinés. Granger et le Régiment du Soleil ne seront pas mortels. Nous sommes disciplinés, nous sommes mortels, nous sommes des Dragons. La bataille est sur le point de commencer et nous sommes sur le point de la gagner.~»

\later

(Discours improvisé donné par le général Potter à la Légion du Chaos, immédiatement avant leur première bataille le 3 novembre 1001 à 14h56~:)

Mes troupes, je ne vais pas vous mentir, notre situation est fort sombre. L'armée du Dragon n'a jamais perdu une bataille. Et Hermione Granger… a une très bonne mémoire. La vérité, c'est que la plupart d'entre vous vont probablement mourir. Mais nous devons gagner. Nous devons gagner pour qu'un jour les enfants puissent à nouveau connaître le goût du chocolat. Tout est en jeu. Littéralement tout. Si nous perdons, l'univers entier claquera comme une ampoule. Et je me rends maintenant compte que la plupart d'entre vous ne savent pas ce qu'est une ampoule. Eh bien, croyez-moi sur parole, nous sommes mal barrés. Mais si nous devons tomber, tombons en combattant, comme des héros, afin que lorsque les ténèbres s'approcheront nous puissions nous dire~: \emph{au moins nous nous sommes amusés}. Avez-vous peur de mourir~? Je sais que moi, oui. Je peux sentir ces frissons glacés de peur comme si quelqu'un injectait de la crème glacée dans mon T-shirt. Mais je sais… que l'Histoire nous regarde. Elle nous regardait lorsque nous nous changions et que nous enfilions nos uniformes. Elle prenait probablement des photos. Et l'Histoire, mes troupes, est écrite par les vainqueurs. Si nous gagnons cette guerre, nous pourrons écrire notre propre Histoire. Une Histoire dans laquelle Poudlard fut fondée par quatre elfes renégats. Nous pourrons faire étudier cette Histoire à tout le monde même si ce n'est pas vrai, et s'ils ne répondent pas correctement aux contrôles… ils redoubleront. Cela ne mérite-t-il pas de se sacrifier~? Non, ne répondez pas. Certaines choses ne doivent pas être sues. Aucun d'entre nous ne sait pourquoi il est là. Aucun d'entre nous ne sait pourquoi il se bat. Nous nous sommes juste réveillés dans ces uniformes, dans cette mystérieuse forêt, ne sachant rien hormis le fait que de gagner est le seul moyen de récupérer nos noms et notre mémoire. Les élèves dans ces autres armées, là, tout autour… ils sont exactement comme nous. Ils ne veulent pas mourir. Ils se battent pour se protéger les uns les autres, pour protéger les derniers amis qui leur restent. Ils se battent parce qu'ils savent qu'ils ont des familles à qui ils manqueront, même s'ils ne peuvent pas s'en souvenir pour l'instant. Ils se battent peut-être même pour sauver le monde. Mais nous avons une meilleure raison de combattre. Nous nous battons parce que nous aimons ça. Nous nous battons pour amuser de bizarres monstruosités venues d'au-delà de l'Espace et du Temps. Nous nous battons parce que nous sommes le Chaos. Bientôt, la bataille finale commencera, alors laissez-moi vous le dire maintenant, car je n'en aurai pas l'occasion plus tard que ça a été un honneur d'être votre commandant, aussi bref que ce fut. Merci, merci à tous. Et souvenez-vous, votre but n'est pas seulement de faucher l'ennemi, notre but est de lui faire peur.

\later

Un grand gong retentit dans la forêt.

Et le Régiment du Soleil se mit en marche.

\later

La tension monta et monta et Harry et les dix-neuf autres soldats qui restaient attendirent que les rapports aériens arrivent. Cela n'aurait pas dû prendre longtemps, les balais étaient rapides et les distances dans la forêt n'étaient pas très grandes…

Deux balais approchèrent à grande vitesse depuis le camp de Drago et tous les soldats se tendirent. Ils n'exécutaient pas les manœuvres qui étaient le code du jour pour signaler un balai \emph{ami}.

«~\emph{Dispersion et feu~!}~» rugit le général Potter, et il accompagna ses mots de gestes, fonçant à pleine vitesse à couvert, dans la forêt~; puis, dès qu'il eut atteint les arbres, il pivota, leva sa baguette, essaya de trouver les balais dans le ciel…
«~C'est bon~! cria une voix. Ils repartent~!~»

Harry haussa mentalement les épaules. Il aurait été impossible d'empêcher Drago d'obtenir cette information, et il avait seulement appris qu'ils se tenaient immobiles.

Et les Chaotiques émergèrent lentement de la forêt…

«~Balais approchants, ils viennent de chez Granger~! cria une autre voix. Je pense que c'est Green Leader, il a fait le chuté vrillé~!~»

Quelques instants plus tard, Theodore Nivett plongeait depuis le ciel et se posait entre les soldats.

«~Granger a divisé ses forces en deux groupes~!~» hurla Nott tandis qu'il planait sur son balai. De la sueur tachait son uniforme et toute réserve avait quitté sa voix. «~Elle attaque les deux armées~! Deux balais couvrent chaque force, ils m'ont poursuivi sur la moitié du chemin~!~»

\emph{Diviser son armée, mais qu'est-ce que…}

Une large force concentrant son feu sur une force plus faible pouvait la réduire à néant sans recevoir beaucoup de dommages en retour. Si vingt soldats faisaient face à dix soldats, vingt sorts seraient dirigés vers dix personnes, avec seulement dix sorts allant dans l'autre sens, donc à moins que chacun de ces dix sorts ne soit les premiers et n'atteignent leur cible, la force la plus faible perdrait plus de personnes qu'elle ne saurait en emporter avec elle. \emph{Vaincue dans le détail} était le terme militaire pour ce qui arrivait lorsque l'on divisait ainsi ses forces. À quoi pouvait donc \emph{penser} Hermione…

Puis Harry comprit.

\emph{Elle se comporte de façon équitable.}

L'année en cours de Défense allait être longue.

«~Très bien, dit Harry d'une voix forte afin que son armée puisse l'entendre. Nous attendrons que l'escadrille Red fasse son rapport, puis nous irons obscurcir un peu de soleil.~»

\later

Drago écouta les rapports des escadrilles en gardant un visage calme, tout le choc masqué à l'intérieur. À quoi pouvait donc \emph{penser} Hermione~?

Puis Drago comprit.

\emph{C'est une feinte}.

L'une des deux forces de Soleil allait changer de direction et les deux convergeraient sur… qui~?

\later

Neville Londubat marchait dans la forêt en direction de la force Soleil approchante, jetant occasionnellement un coup d'œil en l'air pour les balais. À ses côtés marchaient ses camarades d'escouade, Melvin Coote et Lavande Brown de Gryffondor, et Allen Flint de Serpentard. Allen Flint était leur Suggéreur d'escouade bien que Harry ait d'abord dit à Neville, en privé, que le poste était à lui s'il le voulait.

Harry avait dit beaucoup de choses à Neville en privé, à commencer par «~Tu sais, Neville, si tu veux être aussi génial que le Neville imaginaire qui vit dans ta tête mais qui n'a le droit de rien faire parce que tu as peur, tu devrais vraiment t'inscrire aux armées du professeur Quirrell.~»

Neville était \emph{certain} que le Survivant était télépathe. Il était tout bonnement impossible que Harry Potter l'ait appris autrement. Neville n'avait jamais parlé de ça avec \emph{personne} ni donné le moindre signe~; et les \emph{autres} n'étaient pas comme ça, pas qu'il ait pu l'observer en tout cas.

Et la promesse de Harry s'était réalisée, c'\emph{était} différent d'un entraînement en cours de Défense. Neville avait espéré que l'entraînement résolve tout ce qui n'allait pas chez lui, et, eh bien, ça n'avait rien résolu. Même s'il pouvait jeter quelques sorts sur un autre élève pendant le cours tandis que le professeur Quirrell regardait pour s'assurer que tout allait bien, même s'il pouvait éviter et riposter quand c'était \emph{autorisé} et que tout le monde \emph{s'attendait} à ce qu'il le fasse et que les gens le regarderaient bizarrement s'il ne le faisait \emph{pas}, rien de tout cela ne lui avait permis de véritablement s'affirmer.

Mais faire partie d'une \emph{armée}…

Quelque chose d'étrange remuait à l'intérieur de Neville tandis qu'il marchait au pas dans la forêt aux côtés de ses camarades, l'emblème des doigts prêts à claquer cousu sur leur uniforme.

Il avait le droit de marcher normalement s'il le voulait, mais il avait envie d'aller au pas.

À côté de lui, Melvin, Lavande et Allen semblaient tous avoir envie de marcher au pas.

Et Neville commença à doucement fredonner la Chanson du Chaos.

La mélodie était celle qu'un Moldu aurait identifié comme étant la Marche Impériale, aussi connue sous le nom de “Thème de Dark Vador”~; et les mots que Harry avait ajoutés étaient simples à mémoriser.
\vskip 0pt plus 4\baselineskip\settowidth{\versewidth}{Doom doom doom-doom-doom doom doom} \begin{verse}[\versewidth] Doom doom doom\\ Doom doom doom doom doom doom\\ Doom doom doom\\ Doom doom doom doom doom doom\\ \shout{Doom} doom \shout{doom}\\ Doom doom doom-doom-doom doom doom\\ Doom doom-doom-doom doom doom\\ Doom doom doom, doom doom doom. \end{verse}\vskip 0pt plus 4\baselineskip

Au deuxième couplet les autres s'étaient joints à lui, et bientôt on put entendre le même doux chant venir d'autres lieux de la forêt.

Et Neville marchait au côté de camarades Légionnaires du Chaos,\\ d'étranges sentiments remuaient en son cœur,\\ son imagination devenait réalité,\\ et de ses lèvres s'écoulait l'effrayante chanson du destin.

\later

Harry regardait les corps éparpillés dans la forêt. Quelque chose au fond de lui eut la nausée et il dut se rappeler qu'ils étaient seulement endormis. Il y avait des filles parmi ceux tombés, et étrangement cela rendait la chose pire~; il devrait prendre garde de ne pas mentionner cela devant Hermione ou les Aurors retrouveraient ses restes fourrés dans une \emph{petite} théière.

La moitié de l'armée de Soleil n'avait pas donné beaucoup de fil à retordre à Chaos. Les neufs soldats au sol avaient accouru en hurlant des propos incohérents, leur sort de bouclier simple activé, des écrans circulaires destinés à protéger leur visage et leur poitrine. Mais on ne pouvait pas tirer et maintenir le bouclier en même temps, et les soldats de Harry avaient simplement visé les jambes. Tous les Soleil sauf un étaient tombés alors que les “\emph{Somnium}~!” retentissaient. La dernière avait désactivé son bouclier et était parvenue à atteindre l'un des soldats de Harry avait d'être touchée par la seconde vague de sorts de sommeil (le sort de sommeil était sans danger, même à répétition). Les deux balais de Soleil avaient été bien plus difficiles à abattre et s'étaient rendus responsables de la perte de trois chaotiques avant d'être submergés par un feu sol-air nourri.

Hermione n'était pas parmi ceux qui étaient tombés. Drago avait dû l'avoir et cela mettait Harry en \emph{colère} pour une raison qui lui était incompréhensible, il ne savait pas s'il ressentait un devoir de protection envers Hermione ou s'il se sentait floué de ne pas avoir pu l'abattre \emph{lui-même} ou si c'était les deux.

«~Très bien, dit Harry élevant la voix. Mettons-nous tous d'accord sur une chose, ce n'était pas un vrai combat. C'était le général Granger commettant une erreur lors de sa première bataille. Le véritable combat d'aujourd'hui sera avec l'Armée du Dragon et il ne ressemblera en rien à ceci. Ça sera beaucoup plus amusant. Partons d'ici.~»

\later

Un balai s'abattit depuis le ciel, s'approchant effroyablement vite, puis il se cabra et décéléra si rapidement qu'on pouvait presque entendre l'air hurler et protester, et le balai s'immobilisa juste à côté de Drago.

Ce n'était pas de la prise de risque inconsciente. Grégory Goyle était \emph{bon à ce point} et il ne perdait pas de temps.

«~Potter arrive, dit Grégory sans la moindre trace de sa fausse voix traînante habituelle. Ils ont encore leurs quatre balais, tu veux que je les descende~?

--- Non, dit sèchement Drago. Combattre au-dessus de leur armée leur donne un trop grand avantage, ils te tireront dessus depuis le sol et même toi ne pourras peut-être pas les éviter. Attends que toutes les forces se soient engagées dans le combat.~»

Drago avait perdu quatre dragons contre douze soleils. Apparemment le généralement Granger \emph{avait été} aussi incroyablement stupide qu'elle en avait donné l'impression, même si elle n'avait pas fait partie des attaquants, et Drago n'avait donc eu l'opportunité ni de la railler ni de lui demander ce à quoi elle avait bien pu penser.

Comme tous le savaient, la vraie bataille serait contre Harry Potter.

«~Préparez-vous~! rugit Drago à l'intention de ses troupes. Restez avec vos camarades, agissez comme une seule unité, tirez dès que l'ennemi est à portée~!~»

Discipline contre Chaos.

Ça ne devrait pas être bien difficile.

\later

L'adrénaline courait et courait dans le sang de Neville, si bien qu'il avait l'impression de pouvoir à peine respirer.

«~Nous approchons, dit le général Potter d'une voix à peine assez forte pour être entendue de toute l'armée. Il est temps de nous disperser.~»

Les camarades de Neville s'éloignèrent de lui. Ils se soutiendraient toujours, mais si vous restiez amassés, l'ennemi vous frappait bien plus facilement~; un tir visant l'un de vos camarades pourrait manquer et vous toucher à la place. Vous seriez bien plus difficile à atteindre en vous dispersant et en vous déplaçant aussi vite que possible.

La première chose que le général Potter avait faite pendant ses sessions d'entraînement était de les faire se tirer l'un sur l'autre en courant, ou bien avec les deux immobiles, ou bien avec une personne bougeant et l'autre restant immobile -- le contre du sort de sommeil était simple à utiliser, mais on y avait pas droit pendant les vraies batailles. Le général Potter avait minutieusement noté tout ce qui s'était passé puis il avait fait quelques graphiques et quelques équations, et il avait ensuite annoncé qu'il était plus logique de se concentrer non pas sur la lenteur et la précision des tirs mais plutôt sur la vitesse et l'esquive.

Le fait de ne pas marcher au côté de ses camarades embêtait encore Neville, mais les effrayants cris de guerre qu'ils avaient appris tonnaient déjà au-dessus de sa tête et cela compensait son désarroi pour beaucoup.

Cette fois, se jura intérieurement Neville, sa voix n'allait absolument certainement pas se mettre à couiner.

«~Boucliers levés, dit le général Potter, déflecteurs concentrés sur l'avant.

--- \emph{Contego}~», murmura l'armée, et les écrans circulaires apparurent de nulle part devant leur tête et leur poitrine.

Un goût pierreux emplit la bouche de Neville. Le général Potter ne leur aurait pas ordonné de lancer leur bouclier à moins qu'ils ne soient presque à portée. Neville pouvait voir les formes en uniforme des dragons qui se déplaçaient à travers le dense feuillage de la forêt, et les dragons devaient eux aussi les voir…

«~\emph{Attaque~!}~» dit un cri au loin, la voix de Drago Malfoy, et le général Potter mugit~: «~\emph{Chargez…}~»

Toute l'adrénaline du sang de Neville fut libérée, ses jambes prirent le contrôle, l'envoyant voltiger plus vite qu'il n'avait jamais couru, droit vers l'ennemi, sachant sans avoir besoin de regarder que tous ses camarades faisaient de même.

«~\emph{Du sang pour les dieux du sang}~! cria Neville. \emph{Des crânes pour le trône de crâne~!} \emph{Ia~! Shub-Niggurath~! Le flanc de l'ennemi est sur le côté~!}~»

Il y eut un impact muet lorsque les sorts se brisèrent sur le bouclier de Neville. Si d'autres sorts avaient été tirés, ils n'avaient pas atteint leur cible.

Neville vit le bref regard de peur sur le visage de Wayne Hopkins, qui se tenait à côté de deux Gryffondor que Neville ne reconnaissait pas, et alors…

… Neville abaissa son bouclier simple et fit feu sur Wayne…

… rata…

… ses jambes coururent \emph{droit} à travers le groupe ennemi en direction de trois autres dragons, leur baguette se levant vers lui, leur bouche s'ouvrant…

… sans même s'en rendre compte, Neville se laissa chuter sur le sol forestier juste au moment où trois voix crièrent «~\emph{Somnium}~!~»

Cela fit mal, des cailloux et des racines dures s'enfonçant dans Neville tandis qu'il roulait, ce n'était pas aussi douloureux que de tomber de son balai, mais il avait quand même heurté le sol assez durement et soudain une intuition le frappa et il resta immobile et ferma les yeux.

«~Arrêtez ça~! cria une voix. Ne tirez pas, on est des dragons~!~»

Avec un éclair de satisfaction glorieuse, Neville comprit qu'il était parvenu à se placer entre deux groupes de dragons juste au moment où l'un des groupes lui avait tiré dessus. Harry avait parlé de cette tactique, dont le but était que l'ennemi hésite à ouvrir le feu, mais apparemment la méthode marchait encore mieux que cela.

Et en plus de ça, les dragons pensaient qu'ils l'avaient \emph{eu} puisqu'ils l'avaient vu tomber au moment où ils avaient tiré.

Neville compta mentalement jusqu'à vingt puis il ouvrit les yeux un chouilla.

Les trois dragons étaient très proches, leurs têtes pivotant en tous sens tandis que des «~\emph{Somnium~!~»} et «~\emph{Des crânes pour le trône de crânes~!}~» emplissaient l'espace autour d'eux. Tous trois avaient leur bouclier simple levé.

Neville avait toujours sa baguette en main, et il ne lui fallut pas faire un grand effort pour la pointer vers l'une des bottes du garçon et murmurer «~\emph{Somnium}.~»

Neville ferma rapidement les yeux et relaxa sa main lorsqu'il entendit le garçon chuter.

«~\emph{C'est venu d'où~?}~» cria Justin Finch-Fletchley, et Neville entendit des bruissements sur le sol de la forêt alors que les deux dragons se retournaient encore et encore, à la recherche d'un ennemi.

«~\emph{Reformez les rangs}~! mugit la voix de Malfoy. \emph{Revenez à moi, tout le monde, ne les laissez pas vous éparpiller~!}~»

Les oreilles de Neville entendirent les deux dragons sauter au-dessus de son corps allongé et partir en courant.

Neville ouvrit les yeux et se remit assez douloureusement sur pied, puis il réorienta sa baguette et prononça le nouveau sort que le général Potter leur avait enseigné à tous. Ils ne pouvaient pas créer de véritables sorts d'illusion pour troubler l'ennemi, mais même à leur âge ils pouvaient…

«~\emph{Ventriliquo}~», chuchota Neville, pointant sa baguette du côté de Justin et de l'autre garçon, puis il hurla~: «~\emph{Pour Cthulhu et pour la gloire~!}~»

Justin et l'autre garçon s'arrêtèrent abruptement, tournant leur bouclier vers l'endroit d'où Neville avait fait jaillir son cri de guerre, et c'est alors que de multiples «~\emph{Somnium}~!~» jaillirent brusquement et que l'autre garçon s'effondra avant que Neville n'ait pu prendre le temps de viser.

«~\emph{Le dernier est à moi}~!~» hurla Neville, et il commença à sprinter droit vers Justin, qui avait été méchant avec lui jusqu'à ce que des Poufsouffle plus âgés ne lui fassent la morale. Neville était entouré par ses camarades et \emph{ça} voulait dire…

«~\emph{Attaque spéciale, bond chaotique~!}~» cria Neville tout en courant, et il sentit son corps devenir plus léger, puis plus léger encore, au fur et à mesure que ses camarades tournaient leur baguette vers lui et jetaient discrètement le sort de lévitation, et Neville leva sa main gauche et claqua des doigts et utilisa ensuite ses jambes pour se propulser loin du sol aussi fort qu'il le pouvait et il \emph{s'éleva} dans les airs. Une expression de surprise totale apparut sur le visage de Justin lorsque Neville passa \emph{au-dessus} de son bouclier et qu'il dirigea sa baguette vers le bas, en direction de la forme en dessous de lui, et qu'il cria~: «~\emph{Somnium}~!~»

Parce qu'il avait trouvé ça cool, voilà pourquoi.

Neville ne plaça pas correctement ses pieds au moment de l'atterrissage et il laboura pas mal le sol, mais deux des trois autres légionnaires du chaos étaient parvenus à garder leur baguette sur lui pendant toute la durée de son vol et il n'atterrit donc pas trop durement.

Et Neville se remit sur pied, haletant. Il savait qu'il aurait dû bouger, des gens hurlaient «~\emph{Somnium}~!~» un peu partout…

«~\emph{Je suis Neville, le dernier des Londubat}~!~» cria Neville vers le ciel, sa baguette pointée à la verticale comme s'il mettait au défi les cieux d'un bleu éclatant, sachant que rien après ce jour ne serait plus pareil. «~\emph{Neville du Chaos~! Affrontez-moi si vous l'o-}~»

(Lorsque Neville se réveilla, on lui dit que l'Armée du Dragon avait pris sa sortie comme le signal de la contre-attaque).

\later

La fille à côté de Harry s'affaissa, prenant le tir qui lui était destiné, et il put entendre l'exultation lointaine de M. Goyle, puis le balai de ce dernier fonça à côté d'eux, tranchant l'air avec tant de force qu'il aurait dû exploser dans son sillage.

«~\emph{Luminos}~!~» cria l'un des garçons à côté de Harry. Il n'avait pas été capable de récupérer assez de force magique pour le faire plus tôt, et M. Goyle l'esquiva sans ralentir.

Chaos n'avait maintenant plus que six soldats debout et l'Armée du Dragon en avait deux, et le seul problème était que l'un de ces soldats était invincible et qu'il fallait trois soldats ne serait-ce que pour le maintenir sous son bouclier.

Ils avaient perdu plus de soldats à cause de M. Goyle qu'à cause de tous les autres dragons réunis, et il ondulait et esquivait à travers les airs si vite que personne ne pouvait l'atteindre, et il pouvait \emph{tirer sur les gens tout en esquivant}.

Harry avait imaginé toutes sortes de façons d'arrêter M. Goyle mais aucune n'était \emph{sûre}, même utiliser le sort de lévitation pour le ralentir (c'était un rayon continu bien plus facile à diriger) ne serait pas sûr, car il pourrait tomber de son balai~; lui jeter des objets n'était pas sûr, et alors que le sang de Harry commençait à se glacer il lui devenait de plus en plus difficile de se souvenir de ce fait.

\emph{C'est un jeu. Tu n'essaies pas de le} tuer\emph{. Ne gâche pas tous tes plans pour un jeu…}

Harry pouvait voir le motif, il pouvait voir comment M. Goyle ondulait, il pouvait voir comment et quand ils devraient tous tirer pour créer une toile de tirs que M. Goyle ne pourrait pas éviter, mais il n'avait tout simplement pas été capable de \emph{l'expliquer} assez vite à ses soldats, ils ne pouvaient pas coordonner leurs tirs aussi bien que nécessaire, et maintenant ils n'avaient plus assez de gens pour le faire…

\emph{Je refuse de perdre, pas comme ça, pas toute mon armée à cause d'un seul soldat~!}

Le balai de M. Goyle tourna plus rapidement que cela n'aurait dû être possible et il commença alors à se diriger vers Harry et celles de ses troupes qui étaient encore debout, Harry pouvait sentir le garçon à côté de lui devenir nerveux, prêt à se jeter devant son général.

\emph{PAS QUESTION.}

La baguette de Harry se leva, dirigée vers M. Goyle, et l'esprit de Harry visualisa le motif, et les lèvres de Harry s'ouvrirent et sa voix hurla…

«~\emph{-}~»

\later

Lorsque les yeux de Harry s'ouvrirent de nouveau, il était confortablement allongé, mains jointes sur la poitrine, tenant sa baguette tel un héros tombé au combat.

Harry se rassit lentement. Sa \emph{magie} lui faisait mal, une sensation étrange mais pas totalement déplaisante, tout à fait comme la brûlure et la léthargie qui survenaient après un exercice physique difficile.

«~Le général est éveillé~!~» cria une voix. Harry cligna des yeux et mit au point en direction de celle-ci.

Quatre de ses soldats tenaient leur baguette sur un hémisphère prismatique iridescent, et Harry comprit que la bataille n'était pas terminée. Ah oui… il n'avait pas été touché par un sort de sommeil, il s'était juste épuisé, donc il était toujours en jeu en se réveillant.

Harry se doutait que quelqu'un allait lui donner une leçon et lui expliquer qu'il ne fallait pas épuiser sa magie jusqu'à l'inconscience pour un jeu d'enfants. Mais il n'avait pas fait de mal à M. Goyle en se mettant en colère et c'était ça qui comptait.

Puis l'esprit de Harry découvrit que ce qui venait de se produire impliquait autre chose. Il baissa les yeux jusqu'à l'anneau d'acier à l'auriculaire de sa main gauche et faillit jurer à voix haute lorsqu'il vit que le petit diamant manquait et qu'il y avait un marshmallow au sol près de l'endroit où il était tombé.

Il avait maintenu cette métamorphose pendant dix-sept jours et maintenant il lui faudrait recommencer.

Ça aurait pu être pire. Il aurait pu faire ça quatorze jours plus tard, \emph{après} que le professeur McGonagall l'ait autorisé à métamorphoser le rocher de son père. C'était là une très bonne leçon apprise de la moins dangereuse des façons possibles.

\emph{Note à moi-même~: toujours enlever l'anneau de mon doigt avant de vider toute ma magie.}

Harry se releva en poussant au sol avec une certaine difficulté. Utiliser toute la magie n'épuisait pas les muscles, mais courir entre les arbres, si.

Il tituba jusqu'à l'hémisphère iridescent qui contenait Drago Malfoy, lequel tenait sa baguette levée pour maintenir le bouclier et souriait froidement en regardant Harry.

«~Où est le cinquième soldat~? dit Harry.

--- Euh… dit un garçon dont Harry n'arrivait pour l'instant pas à se remémorer le nom. J'ai jeté un sort de sommeil sur le bouclier et il a rebondi et a touché Lavande, je veux dire, le sort n'aurait pas dû rebondir selon cet angle mais ça l'a quand même touché…~»

Drago avait un sourire en coin.

«~Laisse-moi deviner, dit Harry en regardant Drago droit dans les yeux, ces jolis petits trios sont la formation utilisée par les organisations militaires magiques professionnelles~? Composés de soldats entraînés qui peuvent facilement toucher des cibles en mouvement si leurs mains sont stables et qui peuvent combiner leurs capacités défensives tant qu'ils restent ensemble~? Contrairement à \emph{tes} soldats~?~»

Le sourire avait disparu du visage de Drago et il était maintenant dur et lugubre.

«~Tu sais~», dit Harry d'un ton léger, sachant qu'aucun de ses soldats ne comprendrait le véritable message qui allait passer entre eux, «~ça démontre juste que tu devrais toujours remettre en question ce que tes modèles font, demander pourquoi ils font les choses ainsi et si ça a toujours un sens de le faire dans le contexte qui te préoccupe. Au fait, n'oublie pas d'appliquer ce conseil à la vie de tous les jours. Et merci pour les cibles lentes et ramassées.~»

Car Drago avait déjà entendu cette leçon, et, Harry le suspectait, l'avait ignorée, soupçonnant Harry de vouloir éloigner sa loyauté des traditions Sang-Pur. Ce que Harry \emph{voulait}, bien sûr. Mais cet exemple fournirait une excellente excuse, le samedi suivant, pour soutenir la notion que remettre l'autorité en question n'était qu'une technique pratique de la vie de tous les jours. Et Harry mentionnerait aussi les expériences qu'il avait faites, d'abord avec des individus puis avec des groupes, pour vérifier que ses idées concernant l'importance de la vitesse était en effet \emph{correctes}, afin de bien faire rentrer la notion que Drago devait en permanence rester à l'affût d'occasions d'appliquer cette méthode au quotidien.

«~Vous n'avez pas \emph{encore} gagné, général Potter~! gronda Drago. Peut-être que le temps va te manquer et que le professeur Quirrell déclarera un match nul.~»

Une remarque juste et digne d'inquiétude. La guerre s'achèverait lorsque le professeur Quirrell déciderait qu'une armée avait en pratique gagné. Le professeur Quirrell avait expliqué qu'il n'y avait pas de conditions de victoire \emph{formelles} parce qu'alors Harry trouverait un moyen de contourner les règles. Harry devait admettre que c'était vrai.

Et Harry ne pouvait pas blâmer le professeur Quirrell de ne pas avoir déjà mis fin au jeu, car il était plausible que le dernier soldat de l'Armée du Dragon puisse abattre les cinq survivants de la Légion du Chaos.

«~Très bien, dit Harry. Quelqu'un ici sait-il quoi que ce soit au sujet du sort de défense du général Malfoy~?~»

Il apparut que le bouclier de Drago était une version du \emph{Protego} standard qui incluait plusieurs désavantages, le plus important étant que le bouclier ne pouvait pas bouger avec le sorcier.

L'avantage -- ou du point de vue de Harry, le désavantage -- était qu'il était plus simple à apprendre, plus simple à jeter, et bien plus simple à maintenir pendant de longues périodes.

Il leur faudrait marteler le bouclier avec des sorts d'attaquer pour le faire tomber.

Et apparemment, Drago pouvait exercer un certain contrôle sur l'angle de réflexion selon lequel les sorts rebondissaient.

L'idée vint à Harry d'utiliser Wingardium Leviosa pour empiler de lourds rochers sur le bouclier jusqu'à ce que Drago ne puisse plus le maintenir sous une telle pression… mais les rochers pourraient alors tomber et heurter Drago, et blesser le général ennemi pour de vrai ne faisait pas partie des objectifs du jour.

«~Donc, dit Harry. Existe-t-il des sorts spécialisés dans le perçage de bouclier~?~»

Il y en avait.

Harry demanda si l'un de ses soldats en connaissait.

Aucun n'en connaissait.

Dans le bouclier, Drago avait de nouveau son sourire en coin.

Harry demanda s'il existait des sorts d'attaque qui ne rebondiraient \emph{pas}.

Il semblait que les éclairs étaient généralement absorbés plutôt que réfléchis par les boucliers.

… personne ne savait jeter le moindre sort ressemblant de près ou de loin à un éclair.

Drago pouffa.

Harry soupira.

Il déposa sa baguette au sol d'un geste délibéré.

Et il annonça d'un ton assez las qu'il ferait tomber le bouclier de Drago lui-même à l'aide d'une méthode qui demeurerait mystérieuse et que tout le monde devrait tirer sur Drago au moment où le bouclier tomberait.

Les Légionnaires du Chaos eurent l'air nerveux.

Drago avait l'air calme, c'est-à-dire qu'il se contrôlait parfaitement.

Une couverture fine et repliée émergea de la bourse de Harry.

Il s'assit à côté du bouclier iridescent et se passa la couverture sur la tête pour que personne ne puisse voir ce qu'il faisait -- à part Drago, bien sûr.

De la bourse émergèrent une batterie de voiture et une paire de câbles de démarrage.

… ce n'était pas comme s'il avait quitté le monde Moldu pour commencer une nouvelle ère de recherche magique sans emmener avec lui un moyen de générer de l'électricité.

Peu après, les Légionnaires du Chaos entendirent un claquement de doigts suivis d'un crépitement venant de sous la couverture. La luminosité du bouclier augmenta, et ils entendirent la voix de Harry dire~: «~Ne soyez pas distraits s'il vous plaît, restez concentrés sur le général Malfoy.~»

On pouvait voir la tension augmenter sur le visage de Drago, ainsi que la furie, l'agacement et la frustration.

Harry lui sourit et mima de ses lèvres «~\emph{te dirai plus tard}~».

Et c'est alors qu'une spirale d'énergie verte jaillit de la forêt et se fracassa sur le bouclier qui fit un bruit de morceaux de verre frottés les uns contre les autres. Drago tituba.

Dans un mouvement de panique soudain et frénétique, Harry détacha les câbles de démarrage de la batterie et les donna à manger à sa bourse, puis il fourra la batterie elle-même dans la bourse, puis il arracha la couverture, ramassa sa baguette et se leva.

Tous ses soldats étaient toujours là, regardant les environs avec frénésie.

«~\emph{Contego}~», dit Harry, et ses soldats l'imitèrent, mais Harry ne savait même pas dans quelle direction orienter son bouclier. «~Quelqu'un a-t-il vu d'où cela venait~?~» Des têtes se secouèrent. «~Et général Malfoy, cela vous embêterait-il de me dire si \emph{vous} avez eu le général Granger~?

--- Bien sûr, dit Drago d'une voix acide, bien sûr que ça m'embêterait.~»

\emph{Oh bon sang}.

L'esprit de Harry commença à faire des calculs, Drago dans le bouclier, passablement épuisé, Harry épuisé aussi, Hermione dans les bois Dieu savait où, Harry et quatre autres chaotiques encore debout…

«~Vous savez, général Granger, dit Harry haut et fort, vous auriez vraiment dû attendre que j'aie vaincu le général Malfoy avant d'attaquer. Vous auriez peut-être pu abattre \emph{tous} les survivants.~»

De quelque part s'éleva le rire aigu d'une fille.

Harry se figea.

\emph{Ce n'était pas Hermione}.

Et c'est alors que le chant terrifiant, étrange et joyeux s'éleva, venant de tout autour d'eux.

«~\emph{N'aie pas peur ne sois pas triste,\\ On ne fait du mal qu'aux méchants…}

--- \emph{Granger a triché~!} cracha Drago depuis l'intérieur du bouclier. Elle a réveillé ses soldats~! Pourquoi le professeur Quirrell ne…

--- Laisse-moi deviner~», dit Harry, la nausée nouant déjà son estomac. Il détestait vraiment perdre. «~C'était une bataille très facile, non~? Ils sont tous tombés comme des mouches~?

--- Oui, dit Drago. Nous les avons tous eus du premier coup…~»

Le regard de compréhension horrifiée se répandit de Drago jusqu'aux Légionnaires du Chaos.

«~Non, dit Harry, on ne les a pas eus.~»

Des formes en camouflage apparaissaient parmi les arbres.

«~Alliés~? dit Harry.

--- Alliés, dit Drago.

--- Bien~», dit la voix du général Granger, et une spirale d'énergie verte jaillit des bois et réduisit le bouclier de Drago en miettes.

\later

Le général Granger passa le champ de bataille en revue avec un sentiment de satisfaction certain. Il n'y avait plus que neuf Soldats du Soleil mais c'était probablement assez pour s'occuper du dernier survivant ennemi, d'autant plus que Parvati, Anthony et Ernie pointaient déjà leur baguette en direction du général Potter, qu'elle voulait voir pris en vie (enfin, conscient).

C'était Mal, elle le savait, mais elle voulait vraiment vraiment \emph{vraiment} pavoiser.

«~Il y a un truc, c'est ça~?~» dit Harry, la tension perçant dans sa voix. «~Il \emph{doit} y avoir un truc. Tu ne peux pas juste te transformer en un général parfait. Pas en plus de tout le reste. Tu n'es pas Serpentard à ce point~! Tu n'écris pas des poèmes effrayants~! \emph{Personne n'est bon en tout}~!~»

Le général Granger regarda ses Soldats du Soleil puis revint à Harry. Tout le monde regardait probablement cette scène depuis les écrans situés à l'extérieur.

Et le général Granger dit~:

«~Je peux tout faire si j'étudie assez.

--- Oh arrête ça c'est n'im…

--- \emph{Somnium}.~»

Harry s'effondra au sol au milieu de sa phrase.

«~SOLEIL GAGNE~», tonna l'énorme voix du professeur Quirrell, semblant venir à la fois de partout et de nulle part.

«~La gentillesse a gagné~! s'écria le général Granger.

--- \emph{Hourra}~!~» crièrent les Soldats du Soleil. Même les Gryffondor le dirent, et ils le dirent avec fierté.

«~Et quelle est la morale de la bataille d'aujourd'hui~? dit le général Granger.

--- \emph{On peut tout faire si on étudie assez~!}~»

Et les survivants du Régiment du Soleil marchèrent au pas vers le champ de la victoire, chantant leur rengaine en avançant~:

\begin{verse}[\versewidth]
N'aie pas peur ne sois pas triste,\\
On ne fait du mal qu'aux méchants\\
On leur donne un nouveau chez eux\\
Où de nouveaux amis les gardent\\
Sois sûr de dire qui t'envoie,\\
Le Régiment du Soleil Granger~!
\end{verse}

%  LocalWords:  Ia Shub Niggurath os lu bu

\chapter{Gratification différée}

\iftoggle{embedepigraphs}{Du sang pour le dieu du sang~! Des crânes pour J. K. Rowling~!

\iftoggle{embedepigraphs}{\newpage}{}
}{}

\lettrine{D}{rago} arborait une expression sévère, et sa robe à bordures vertes semblait être plus solennelle, bien plus sérieuse et bien mieux coupée que celles portées par les deux garçons derrière lui.

«~Parle, dit Drago.

--- Ouais~! Parle~!

--- T'as entendu l'boss~! Parle~!

--- Vous deux, en revanche, \emph{fermez-la}.~»

Le dernier cours de vendredi allait commencer dans le vaste auditorium où les quatre Maisons apprenaient la Défense… euh, la Magie de Combat.

Le dernier cours du vendredi.

Harry espérait que ce cours ne serait pas stressant, et que le génial professeur Quirrell se rendrait compte que ce n'était peut-être pas le meilleur moment pour attirer l'attention sur Harry. Il avait un peu récupéré, mais…

… mais juste au cas où, il valait probablement mieux s'adonner à un petit moment de détente.

Harry s'enfonça dans sa chaise et octroya un regard d'une grande solennité à Drago et à ses laquais.

«~Vous demandez notre but~? déclama Harry. Je puis répondre par un mot. C'est la victoire. La victoire à tout prix -- la victoire au prix de toutes les terreurs -- la victoire, aussi longue et difficile que puisse être la route, car sans victoire il n'est point de…

--- \emph{Parle-moi de Rogue}, siffla Drago. \emph{Qu'est-ce que tu as fait~?}~»

Harry se débarrassa de sa fausse solennité et jeta un regard plus sérieux à Drago.

«~Tu l'as vu, dit Harry. Tout le monde l'a vu. J'ai claqué des doigts.

--- \emph{Harry~! Arrête de me taquiner~!}~»

Ah, alors il avait été promu à \emph{Harry} maintenant. Intéressant. Et de fait, Harry était certain qu'il était censé s'en rendre compte et se sentir mal s'il n'y répondait pas d'une façon ou d'une autre…

Harry se tapota les oreilles et jeta un regard lourd de sens en direction des laquais.

«~Ils ne parleront pas, dit Drago.

---Drago, dit Harry, je vais être cent pour cent honnête et te dire que, hier, je n'ai pas été particulièrement impressionné par la fourberie de M. Goyle.~»

M. Goyle fit la grimace.

«~Moi non plus, dit Drago. Je lui ai expliqué que j'avais fini par te devoir une faveur à cause de ça.~» (M. Goyle fit de nouveau la grimace) «~Mais il y \emph{a} une grande différence entre ce genre d'erreur et l'indiscrétion. On les a vraiment entraînés à comprendre ça depuis leur enfance.

--- Alors très bien~», dit Harry. Il baissa la voix même si le bruit ambiant était déjà devenu flou en présence de Drago. «~Je suis parvenu à déduire l'un des secrets de Severus et je lui ai fait un peu de chantage.~»

Le visage de Drago se durcit. «~Bon, maintenant dis-moi quelque chose que tu n'as pas confié en stricte confidence aux idiots de Gryffondor, c'est-à-dire l'histoire que tu \emph{voulais} voir répandue dans l'école.~»

Harry sourit involontairement, et il sut que Drago l'avait remarqué.

«~Que dit Severus~? dit Harry.

--- Qu'il ne s'était pas rendu compte à quel point les émotions des jeunes enfants étaient sensibles, dit Drago. Même aux Serpentard~! Même à \emph{moi}~!

--- Es-tu certain, dit Harry, de vouloir savoir quelque chose que même le Directeur de ta Maison préférerait te voir ignorer~?

--- Oui~», dit Drago sans hésitation.

\emph{Intéressant}. «~Alors tu vas vraiment commencer par renvoyer tes laquais, parce que je ne suis pas certain de pouvoir croire tout ce que tu dis à leur sujet.~»

Drago hocha la tête. «~D'accord.~»

M. Crabbe et M. Goyle avaient l'air \emph{vraiment} mécontents. «~Patron…~» dit M. Crabbe.

«~Vous n'avez donné aucune raison à M. Potter de vous faire confiance, dit Drago. Partez~!~»

Ils s'en furent.

«~En particulier~», dit Harry, baissant encore plus la voix, «~je ne suis pas \emph{entièrement} certain qu'ils ne rapporteraient pas ce que je dis à Lucius.

--- Père ne \emph{ferait} pas ça~!~» dit Drago, l'air vraiment horrifié. «~Ils sont à \emph{moi}~!

--- Je suis désolé, Drago, dit Harry. Je ne suis pas certain de pouvoir croire tout ce que tu crois au sujet de ton père. Imagine que c'était ton secret, et que je te disais que mon père ne ferait pas une chose pareille.~»

Drago hocha lentement la tête. «~Tu as raison. \emph{Je suis} désolé, Harry. J'avais tort de te demander ça.~»

\emph{Comment ai-je fait pour être autant promu~? Ne devrait-il pas me détester maintenant~?} Harry avait l'impression qu'il était en présence d'une situation exploitable… il aurait seulement aimé que son cerveau ne soit pas si épuisé. En temps normal, il aurait adoré s'essayer au tissage de complexes intrigues.

«~Bref, dit Harry. Échange. Je te donne une information qui ne circule pas déjà, et qui \emph{n'entre} pas en circulation, et en \emph{particulier} qui ne va pas jusqu'à ton père, et en retour tu me dis ce que Serpentard et toi pensez de toute cette histoire.

--- Marché conclu~!~»

Et maintenant pour rendre ça aussi vague que possible… quelque chose qui ne poserait pas de problème même si ça se savait…

«~Ce que j'ai dit était vrai. J'ai découvert l'un des secrets de Severus, et j'ai fait un peu de chantage. Mais Severus n'était pas la seule personne impliquée.

--- \emph{Je le savais}~!~» dit Drago, exultant.

L'estomac de Harry se serra. Il avait apparemment dit quelque chose de très important et il ne savait pas pourquoi. Ce n'était pas bon signe.

«~Très bien~», dit Drago. Il avait maintenant un grand sourire sur le visage. «~Alors voilà à quoi ressemblait la réaction à Serpentard. D'abord, tous les idiots étaient là, “On déteste Harry Potter~! Allons lui mettre une raclée~!”~»

Harry s'étouffa.

«~Qu'est ce qui \emph{cloche} avec le Choixpeau Magique~? Ce n'est pas du Serpentard, c'est du \emph{Gryffondor}…

--- Tous les enfants ne sont pas des prodiges~», dit Drago, mais il souriait d'un air vilain et conspirateur comme pour sous-entendre qu'en son for intérieur il était d'accord avec Harry. «~Et ça a pris à peu près quinze secondes pour leur expliquer en quoi ce ne serait pas vraiment faire une faveur au professeur Rogue, donc ne t'en fais pas. Bref, après ça il y a eu la deuxième vague d'idiots, ceux qui disaient “On dirait que Harry n'est qu'un bien-pensant de plus après tout.”
--- Et ensuite~?~» dit Harry en souriant, même s'il ne savait absolument pas pourquoi \emph{cette idée} était stupide.

«~Et alors les gens vraiment intelligents ont commencé à parler. Il est évident que tu as trouvé un moyen de mettre \emph{beaucoup} de pression sur Rogue. Et même si ça pourrait être plus d'une chose à la fois… l'idée évidente \emph{suivante} a été que ça avait quelque chose à voir avec l'emprise inconnue que Rogue a sur Dumbledore. J'ai raison~?

--- Pas de commentaire~», dit Harry. Au moins son cerveau comprenait cette partie correctement. La Maison Serpentard \emph{s'était} demandée pourquoi Severus ne s'était pas fait renvoyer. Et ils en avaient conclu que Severus faisait chanter Dumbledore. Cela pouvait-il être vrai~? Mais Dumbledore n'avait pas semblé agir comme si c'était le cas…

Drago continua de parler. «~Et ce que les gens intelligents ont \emph{ensuite} fait remarquer, c'était que si tu pouvais mettre assez de pression sur Rogue pour le forcer à laisser tranquille la moitié de Poudlard, alors ça voulait dire que tu avais probablement assez de pouvoir pour te débarrasser entièrement de lui si tu le voulais. Ce que tu lui as fait subir, c'est une humiliation, tout comme il a essayé de t'humilier -- mais tu nous as laissé notre Directeur de Maison.~»

Harry laissa son sourire grandir.

«~Et alors les gens \emph{vraiment} intelligents,~» dit Drago, le visage maintenant sérieux, «~sont partis, et ils ont eu une petite discussion privée juste entre eux, et quelqu'un a fait remarquer que ce serait très stupide de laisser un ennemi dans les parages comme ça. Si tu avais pu briser son emprise sur Dumbledore, ça aurait été la chose à faire. Dumbledore ferait alors dégager Rogue de Poudlard, et il essaierait peut-être même de le faire abattre, et il te serait \emph{très} reconnaissant, et tu n'aurais pas à t'inquiéter de voir Rogue se faufiler la nuit dans ton dortoir avec des potions intéressantes.~»

Le visage de Harry était à présent neutre. Il n'avait pas pensé à ça et il aurait vraiment, vraiment dû.

«~Et de ça tu as conclu…?

--- L'emprise de Rogue vient d'un secret de Dumbledore et \emph{tu as ce secret}~! Drago exultait. Il ne peut pas être assez puissant pour totalement détruire Dumbledore, sinon Rogue l'aurait déjà utilisé. Rogue refuse d'utiliser son emprise pour autre chose que de rester roi de Serpentard, et même là il n'obtient pas toujours ce qu'il veut, alors ça doit avoir ses limites. Mais ça \emph{doit} être un sacrément bon secret~! Père essaie de le tirer du nez de Rogue depuis des \emph{années}~!

--- Et, dit Harry, maintenant Lucius pense que \emph{je} peux le lui dire. As-tu déjà envoyé une chouette…

--- Je le ferai ce soir~», dit Drago, et il rit. «~La réponse dira~», sa voix prit une cadence différente, plus formelle, «~\emph{Mon fils bien-aimé~: je t'ai déjà parlé de l'importance potentielle de Harry Potter. Comme tu t'en es déjà rendu compte, celle-ci est à présent devenue plus grande et plus urgente. Si tu vois le moindre moyen menant à l'amitié ou le moindre point de pression exerçable, tu dois t'y aventurer, et toutes les ressources des Malfoy seront à ta disposition si besoin est.}~»

Eh ben.

«~Alors, dit Harry, sans commenter sur la véracité de l'édifice compliqué qu'est ta théorie, laisse-moi juste dire que nous ne sommes pas encore de si bons amis que ça.

--- Je sais~», dit Drago. Puis son visage devint \emph{très} sérieux, et sa voix fut difficile à entendre, même dans ce flou sonore. «~Harry, t'est-il venu à l'esprit que si tu sais quelque chose que Dumbledore ne souhaite pas que tu saches, Dumbledore peut simplement te faire exécuter~? Et ça ferait aussi passer le Survivant du statut de meneur concurrent potentiel à celui de martyr de valeur.

--- Pas de commentaire~», dit de nouveau Harry. Il n'avait pas pensé à ça non plus. Ça ne \emph{ressemblait} pas au style de Dumbledore… mais…

«~Harry, dit Drago, tu as clairement un talent \emph{incroyable}, mais tu n'as aucun entraînement et pas de mentors et tu fais parfois des choses stupides et \emph{tu as vraiment besoin d'un conseiller qui sait s'y prendre ou tu vas te faire mal~!}~» Le visage de Drago était plein de ferveur.

«~Ah, dit Harry. Un conseiller, comme Lucius~?

--- Comme \emph{moi}~! dit Drago. Je promets de garder tes secrets à l'abri de Père, à l'abri de \emph{tout le monde}, je t'aiderai juste à accomplir ce que tu veux accomplir~!~»

Wow.

Harry vit zombie-Quirrell passer les portes de la salle en titubant.

«~Le cours est sur le point de commencer, dit Drago, c'est trop tôt. Tu vois~? Je te donne des bons conseils même si ça me dessert. Mais on devrait peut-être \emph{se dépêcher} de devenir des amis plus proches.

--- Je suis ouvert à ça~», dit Harry, qui essayait déjà de trouver un moyen d'exploiter cette possibilité.

«~Un autre conseil~», dit Drago en vitesse tandis que Quirrell traînait des pieds jusqu'à son bureau, «~pour l'instant tout le monde à Serpentard se pose des questions à ton sujet, alors si tu nous fais la cour, comme je pense que tu es en train de le faire, tu devrais faire quelque chose qui signale ton amitié à Serpentard. \emph{Bientôt}, du genre aujourd'hui ou demain.

--- Laisser Severus continuer de donner des points supplémentaires à Serpentard n'a pas suffi~?~» Aucune raison pour que Harry ne s'attribue pas le mérite de ça.

Les yeux de Drago tressaillirent au moment où il comprit, puis il dit rapidement~:

«~Ce n'est pas pareil, crois-moi, ça doit être quelque chose d'évident. Pousse ta rivale sang-de-bourbe Granger dans un mur ou quelque chose du style, tout le monde à Serpentard comprendra ce que ça veut dire…

--- Ce n'est \emph{pas} comme ça que Serdaigle fonctionne, Drago~! Si tu dois pousser quelqu'un contre un mur, ça veut dire que ton cerveau est trop \emph{faible} pour le vaincre comme il se doit, et tout le monde à Serdaigle \emph{sait ça}…~»

L'écran sur le pupitre de Harry s'alluma en clignotant, provoquant une montée soudaine de nostalgie pour la télévision et les ordinateurs.

«~Ahem~», dit la voix du professeur Quirrell, qui semblait s'adresser personnellement à Harry depuis l'écran. «~Merci de vous asseoir.~»

\later

Et les enfants furent tous assis, regardant les écrans-relais sur les pupitres, ou directement vers la grande plate-forme de marbre blanc où se tenait le professeur Quirrell, penché sur son bureau, sur la petite estrade de marbre plus sombre.

«~Aujourd'hui, dit le professeur Quirrell, j'avais prévu de vous enseigner votre premier sort défensif, un petit bouclier qui était l'ancêtre du \emph{Protego} moderne. Mais à la réflexion, et à la lumière des événements récents, j'ai décidé de changer la leçon d'aujourd'hui.~»

Le regard du professeur Quirrell parcourut la rangée de sièges. Harry grimaça depuis son siège dans la dernière rangée. Il avait l'impression de savoir qui allait être appelé.

«~Drago, de la Noble et Ancienne Maison de Malfoy~», dit le professeur Quirrell.

Pfiou.

«~Oui, professeur~?~» dit Drago. Sa voix était amplifiée et semblait venir de l'écran-relais sur le pupitre de Harry qui montrait le visage de Drago en train de parler. Puis l'écran revint au professeur Quirrell, qui dit~:

«~Est-ce votre ambition que de devenir le prochain Seigneur des Ténèbres~?

--- C'est une drôle de question, professeur, dit Drago. Je veux dire, qui serait assez stupide pour admettre ça~?~»

Quelques étudiants rirent, mais pas beaucoup.

«~En effet, dit le professeur Quirrell, et même s'il est inutile de vous poser la question, je ne serais pas le moins du monde surpris s'il y avait un étudiant ou deux dans mes cours qui entretenaient l'ambition de devenir le prochain Seigneur des Ténèbres. Après tout, \emph{je} voulais être le prochain Seigneur des Ténèbres quand \emph{j'}étais un jeune Serpentard.~»

Cette fois le rire fut bien plus général.

«~C'\emph{est} la Maison des ambitions, après tout, dit le professeur Quirrell en souriant. Je ne me suis rendu compte que plus tard que ce que j'aimais vraiment, c'était la Magie de Combat, et que ma véritable ambition était de devenir un grand sorcier combattant et d'enseigner un jour à Poudlard. Quoi qu'il en soit, quand j'avais treize ans, j'ai lu toutes les sections historiques de la bibliothèque de Poudlard en examinant minutieusement les vies et les destins des Seigneurs des Ténèbres passés, et j'ai fait une liste de toutes les erreurs que \emph{je} ne ferai jamais quand \emph{je} serai un Seigneur des Ténèbres…~»

Harry gloussa avant de pouvoir s'en empêcher.

«~Oui M. Potter, très amusant. Alors dites-moi, pouvez-vous deviner quel était le tout premier élément de cette liste~?~»

\emph{Génial}.

«~Euh… ne jamais utiliser une méthode compliquée de s'occuper d'un ennemi quand on peut juste l'Abracadabrer~?

--- Le \emph{terme}, M. Potter, est \emph{Avada Kedavra}~», pour une raison inconnue, la voix du professeur Quirrell était acerbe, «~et non, ce n'était \emph{pas} sur la liste que j'ai faite à l'âge de treize ans. Voudriez-vous réessayer~?

--- Ah… ne jamais se vanter de son plan maléfique auprès de quelqu'un~?~»

Le professeur Quirrell rit. «~Ah, \emph{ça} c'était le numéro deux. Dites-moi, M. Potter, aurions-nous été lire certains livres~?~»

Il y eut plus de rires, avec une nuance de nervosité. Harry serra fermement sa mâchoire et ne dit rien. Nier n'aurait eu aucun effet.

«~Mais non. Le \emph{premier} élément était~: “Je ne m'amuserai pas à provoquer des ennemis puissants et brutaux.” L'histoire du monde serait très différente si Mornelithe Falconsbane ou Hitler avaient appréhendé ce conseil élémentaire. Maintenant, \emph{si}, M. Potter, -- juste \emph{si} vous vous trouviez par hasard entretenir une ambition similaire à celle que j'entretenais lorsque j'étais un jeune Serpentard -- même alors, j'espère que ce ne serait pas votre ambition que de devenir un Seigneur des Ténèbres \emph{stupide}.

--- Professeur Quirrell, dit Harry en serrant les dents, je suis un \emph{Serdaigle}, et ce n'est pas dans mes ambitions que de devenir stupide, point. Je sais que ce que j'ai fait aujourd'hui était bête. Mais ce n'était pas \emph{Ténébreux}~! Ce n'est \emph{pas} moi qui ai porté le premier coup dans ce combat~!

--- M. Potter, vous êtes un idiot. Mais je l'étais moi-même à votre âge. Aussi j'ai anticipé votre réponse et ai altéré la leçon du jour en fonction. M. Gregory Goyle, voudriez-vous vous avancer s'il vous plaît~?~»

Il y eut une pause de surprise dans la classe. Harry ne s'était pas attendu à ça.

Et vu son air, pas plus que M. Goyle, qui avait l'air plutôt hésitant et inquiet tandis qu'il montait sur la plate-forme de marbre et s'approchait de l'estrade.

Le professeur Quirrell se redressa au-dessus de son bureau sur lequel il s'était appuyé. Il eut soudain l'air plus fort, ses mains se resserrèrent en des poings et il adopta une posture martiale clairement reconnaissable.

À cette vue, les yeux de Harry s'écarquillèrent, et il comprit pourquoi M. Goyle avait été appelé.

«~La plupart des sorciers, dit le professeur Quirrell, ne s'embêtent pas beaucoup avec ce qu'un Moldu appellerait les arts martiaux. Une baguette n'est-elle pas plus forte qu'un poing~? Cette attitude est idiote. Les baguettes sont tenues par des poings. Si vous voulez être un grand sorcier combattant, vous \emph{devez} apprendre les arts martiaux jusqu'à un niveau qui impressionnerait même un Moldu. Je vais maintenant démontrer une technique d'une importance vitale, que j'ai apprise dans un \emph{dojo}, une école moldue d'arts martiaux, et dont je parlerai très bientôt. Pour le moment…~» Le professeur Quirrell fit quelques pas, toujours en posture, et s'avança vers l'endroit où se tenait M. Goyle. «~M. Goyle, je vais vous demander de m'attaquer.

--- Professeur Quirrell~», dit M. Goyle, sa voix maintenant autant amplifiée que celle du professeur, «~puis-je vous demander quel niveau…

--- Sixième \emph{dan}. Vous ne serez pas blessé, et moi non plus. Et si vous voyez une ouverture, merci de la prendre.~»

M. Goyle hocha la tête, l'air très soulagé.

«~Notez, dit le professeur Quirrell, que M. Goyle avait peur d'attaquer quelqu'un qui ne connaissait pas les arts martiaux à un niveau suffisant de peur que lui ou moi ne soyons blessés. L'attitude de M. Goyle est exactement la bonne, et il a gagné trois points Quirrell pour cela. Maintenant, battez-vous~!~»

Le jeune garçon se jeta en avant, poings volants en tous sens, et le professeur arrêta chaque coup en dansant à reculons, puis il donna un coup de pied et Goyle bloqua et pivota et essaya de faire chuter Quirrell d'un balayage de la jambe et Quirrell bondit au-dessus et tout allait trop vite pour que Harry comprenne ce qui se passait et soudain Goyle était sur le dos et poussait avec ses jambes et Quirrell \emph{volait} réellement \emph{dans les airs} puis il toucha le sol épaule en avant et fit une roulade.

«~Stop~!~», s'écria le professeur Quirrell depuis le sol, l'air soudain un peu paniqué. «~Vous avez gagné~!~»

M. Goyle s'arrêta si brusquement qu'il vacilla, trébucha, et tomba presque sous l'effet de l'accélération interrompue provoquée par sa charge tête baissée vers le professeur Quirrell. Son visage exprimait un choc intense.

Le professeur Quirrell courba son dos et bondit sur ses pieds grâce à une sorte de saut bizarre qui ne faisait pas usage de ses mains.

Il y eut un silence dans la classe, un silence né d'une confusion totale.

«~M. Goyle, dit le professeur Quirrell, quelle technique vitale ai-je démontrée~?

--- Comment tomber correctement quand quelqu'un vous projette, dit M. Goyle. C'est une des premières leçons qu'on apprend…

--- Ça aussi~», dit le professeur Quirrell.

Il y eut une pause.

«~La technique vitale que j'ai démontrée, dit le professeur Quirrell, est comment perdre. Vous pouvez vous rasseoir, M. Goyle, merci.~»

M. Goyle descendit de la plate-forme, l'air plutôt abasourdi. Harry partageait son sentiment.

Le professeur Quirrell retourna à son bureau et recommença à s'appuyer dessus. «~Nous oublions parfois les choses les plus élémentaires parce que nous les avons apprises il y a trop longtemps. Je me suis rendu compte que j'avais fait de même avec le plan de mon cours. On n'enseigne pas aux étudiants à projeter avant de leur avoir appris à tomber. Et je ne dois pas vous apprendre à vous battre tant que vous ne savez pas perdre.~»

Le visage du professeur Quirrell se durcit et Harry pensa avoir vu dans ses yeux comme une trace de douleur, comme un soupçon de tristesse. «~J'ai appris comment perdre dans un \emph{dojo} en Asie, où vivent, comme tout Moldu le sait, tous les bons pratiquants d'arts martiaux. Ce \emph{dojo} enseignait un style qui avait, auprès des sorciers combattants, la réputation d'être bien adapté aux duels magiques. Le maître du \emph{dojo} -- un vieil homme selon les standards moldus -- était le plus grand maître de cette technique. Il n'avait pas la moindre idée que la magie existait, bien sûr. Je me suis inscrit pour étudier là, et je fus l'un des rares étudiants à être acceptés cette année parmi tous les candidats. Il se peut qu'une touche d'influence spéciale y ait été pour quelque chose.~»

Il y eut des rires dans la salle. Harry ne partagea pas l'hilarité. Ça n'était pas du tout acceptable.

«~Quoi qu'il en soit. Lors de l'un de mes premiers combats à mains nues, après avoir été vaincu d'une façon particulièrement humiliante, j'ai perdu le contrôle de moi-même et j'ai attaqué mon partenaire de lutte…~»

\emph{Berk.}

«~… heureusement avec mes poings plutôt qu'avec ma magie. De façon surprenante, le Maître ne m'exclut pas immédiatement. Mais il me dit qu'il y avait une faille dans mon tempérament. Il me l'expliqua, et je sus qu'il disait vrai. Et il dit alors que je devrais apprendre à perdre.~»

Le visage du professeur Quirrell était vide de toute expression.

«~Sous son ordre direct, tous les étudiants du \emph{dojo} formèrent une ligne. Un par un, ils m'approchèrent. Je ne devais \emph{pas} me défendre. Je devais seulement implorer leur grâce. Un par un, ils me giflèrent, ou ils me frappèrent, et ils me firent tomber au sol. Certains d'entre eux crachèrent sur moi. Ils me traitèrent de tous les noms dans leurs langues natales. Et à chacun je dus répondre, “J'ai perdu~!” et autres déclarations similaires, comme “Arrête-toi, je t'en supplie~!” et “J'admets que tu es meilleur que moi~!”~»

Harry essayait d'imaginer ça et n'y parvenait tout simplement pas. Il était impossible qu'une chose pareille soit arrivée au digne professeur Quirrell.

«~J'étais un prodige de magie de combat, même à l'époque. J'aurais pu tous les tuer avec de la magie sans baguette. Mais je ne l'ai pas fait. J'ai appris à perdre. Aujourd'hui encore, je m'en souviens comme des heures les plus déplaisantes de mon existence. Et lorsque j'ai quitté le dojo huit mois plus tard -- ce qui était beaucoup trop tôt, mais c'était tout le temps que je pouvais me permettre d'y passer -- le Maître me dit qu'il espérait que je comprenais pourquoi cela avait été nécessaire. Et je lui ai dit que c'était une des leçons les plus précieuses que j'avais jamais apprises. Ce qui était vrai, et l'est toujours.~»

Le professeur Quirrell adopta une expression amère. «~Vous vous demandez où cet extraordinaire \emph{dojo} se trouve, et si vous pouvez y étudier. Vous ne le pouvez pas. Car peu de temps après, un autre étudiant en puissance parvint en ce lieu caché, sur cette montagne reculée. Celui-Dont-On-Ne-Doit-Pas-Prononcer-Le-Nom.~»

Il y eut le son de plusieurs grandes inspirations simultanées. Harry se sentit malade. Il savait ce qui allait suivre.

«~Le Seigneur des Ténèbres entra dans l'école à découvert, sans déguisement, les yeux rouges et compagnie. Les étudiants tentèrent de lui bloquer la route, et il transplana simplement à travers eux. Ils étaient terrifiés, mais aussi disciplinés, et le Maître s'avança. Et le Seigneur des Ténèbres exigea -- il ne demanda pas, il exigea -- d'apprendre.~»

Le visage du professeur Quirrell se durcit énormément. «~Peut-être le Maître avait-il lu trop de livres répétant le mensonge selon lequel un véritable pratiquant des arts martiaux pouvait vaincre les démons eux-mêmes. Pour cette raison ou une autre, le Maître refusa. Le Seigneur des Ténèbres lui demanda pourquoi il ne pouvait pas devenir un étudiant. Le Maître lui dit qu'il n'avait aucune patience, et c'est là que le Seigneur des Ténèbres lui arracha la langue.~»

Il y eut un bruit d'étranglement collectif.

«~Vous pouvez deviner ce qui se passa ensuite. Les étudiants essayèrent de se jeter sur le Seigneur des Ténèbres et tombèrent, stupéfixés avant d'avoir pu bouger. Et alors…~»

La voix du professeur Quirrell faiblit pendant un moment, puis il reprit.

«~Il existe un Sortilège Impardonnable, le sort Doloris, qui provoque une insupportable douleur. S'il est maintenu plus de quelques minutes, il crée un état de folie permanent. Un par un, le Seigneur des Ténèbres Endolorit les étudiants du Maître jusqu'à la folie, puis il les acheva d'un Sort de Mort tandis que le Maître était forcé de regarder. Une fois que tous ses étudiants eurent ainsi été tués, le Maître les suivit. J'ai appris cela de la bouche du seul survivant, que le Seigneur des Ténèbres a gardé en vie afin qu'il raconte cette histoire, et qui avait été mon ami…~»

Le professeur Quirrell se détourna, et lorsqu'il refit face à la salle un instant plus tard, il semblait de nouveau calme et composé.

«~Les Seigneurs des Ténèbres ne peuvent pas contrôler leur colère, dit doucement le professeur Quirrell. C'est un défaut quasiment universel chez cette espèce, et quiconque a l'habitude de les combattre apprend bien vite à compter dessus. Comprenez que le Seigneur des Ténèbres ne gagna \emph{pas}, ce jour-là. Son but était d'apprendre les arts martiaux, et pourtant il s'en fut sans avoir eu une seule leçon. Le Seigneur des Ténèbres fut sot de souhaiter que cette histoire soit contée. Car elle ne montre pas sa force, mais plutôt une faiblesse exploitable.~»

Le regard du professeur Quirrell se concentra sur un élève de la classe.

«~Harry Potter, dit le professeur Quirrell.

--- Oui~», dit Harry, la voix rauque.

«~Qu'avez-vous fait de mal aujourd'hui, \emph{précisément}~?~»

Harry crut qu'il allait vomir.

«~J'ai perdu le contrôle de ma colère.

--- Ce n'est \emph{pas} précis, dit le professeur Quirrell. Je le décrirai avec plus d'exactitude. Il existe de nombreux animaux ayant ce qu'on appelle des luttes de dominance. Ils se foncent dessus, cornes en avant -- essayant de s'assommer l'un l'autre, pas de s'encorner. Ils se battent avec leurs pattes -- avec les griffes rétractées. Mais pourquoi avec leurs griffes rétractées~? Ils auraient certainement une meilleure chance de gagner s'ils utilisaient leurs griffes. Mais alors leur ennemi dégainerait peut-être lui aussi ses griffes, et alors, au lieu de résoudre leur lutte de dominance et d'avoir un gagnant et un perdant, ils pourraient tous deux se blesser sérieusement.~»

Le professeur Quirrell sembla regarder droit vers Harry depuis l'écran-relais.

«~Ce que vous avez démontré aujourd'hui, M. Potter, c'est que -- à la différence des animaux qui gardent leurs griffes rétractées et acceptent l'issue du combat -- vous ne savez pas perdre une lutte de dominance. Lorsqu'un \emph{professeur de Poudlard} vous a défié, vous n'avez pas battu en retraite. Lorsqu'il a semblé que vous risquiez de perdre, vous avez dégainé vos griffes, négligeant le danger. Vous avez \emph{renchéri}, puis vous avez \emph{encore} renchéri. Ça a commencé par une gifle donnée par le professeur Rogue, qui était évidemment dominant. Au lieu de perdre, vous avez répondu par une gifle et avez fait perdre dix points à Serdaigle. Très vite vous parliez de quitter Poudlard. Le fait que vous ayez encore plus renchéri dans une direction inconnue et que vous soyez parvenu à gagner ne change rien au fait que vous êtes un idiot.

--- Je comprends,~» dit Harry. Il avait la gorge sèche. Cela \emph{avait} été précis. \emph{Effroyablement} précis. Maintenant que le professeur Quirrell l'avait dit, Harry pouvait voir, rétrospectivement, que c'était une description \emph{exacte} de ce qui s'était passé. Lorsque quelqu'un avait un modèle de vous aussi précis que ça, vous deviez commencer à vous demander s'ils avaient raison sur d'autres sujets, comme sur votre intention de tuer par exemple.

«~M. Potter, la \emph{prochaine} fois que, lors d'un duel, vous renchérissez au lieu de perdre, vous pourriez perdre \emph{tout} ce que vous avez misé. Je ne peux pas deviner ce que vos mises étaient aujourd'hui. Je peux deviner qu'elles étaient bien, bien trop élevées pour valoir la perte de dix points.~»

Comme le destin de l'Angleterre magique, par exemple. C'était ce qu'il avait misé.

«~Vous allez vous défendre en disant que vous essayiez d'aider tout Poudlard, un but bien plus important, méritant qu'on prenne de grands risques. C'est un \emph{mensonge}. Si vous aviez…

--- J'aurais reçu la gifle, attendu, et choisi le meilleur moment pour agir,~» dit Harry d'une voix à nouveau rauque. «~Mais alors j'aurais \emph{perdu}. Je l'aurais laissé me dominer. C'est ce que le Seigneur des Ténèbres n'a pas pu faire face au Maître dont il désirait l'enseignement.~»

Le professeur Quirrell hocha la tête.

«~Je vois que vous avez parfaitement compris. Et donc, M. Potter, aujourd'hui, vous allez apprendre à perdre.

--- Je…

--- Je n'écouterai aucune de vos objections, M. Potter. Il est à la fois évident que vous en avez besoin et que vous êtes assez fort pour le supporter. Je vous assure que l'expérience ne sera pas aussi brutale que celle que j'ai traversée, bien qu'il soit fort probable que vous vous rappeliez ensuite ces quinze minutes comme les pires de votre jeune existence.~»

Harry déglutit.

«~Professeur Quirrell, dit-il d'une petite voix, pourrions-nous faire ça une autre fois~?

--- Non, dit simplement le professeur Quirrell. Vous en êtes au cinquième jour de votre éducation à Poudlard et ceci s'est déjà produit. Nous sommes aujourd'hui vendredi. Notre \emph{prochain} cours de Défense est mercredi. Samedi, dimanche, lundi, mardi, mercredi… Non, nous n'avons \emph{pas} le temps d'attendre.~»

Il y eut quelques rires, mais très peu.

«~S'il vous plaît M. Potter, considérez cela comme un ordre de votre professeur. Je tiens à préciser qu'autrement, je ne vous enseignerai aucun sort offensif, parce que j'apprendrais alors que vous avez sévèrement blessé voir même tué quelqu'un. Malheureusement, j'ai entendu dire que vos doigts sont déjà des armes puissantes. Ne les claquez à aucun moment pendant cette leçon.~»

Plus de rires épars, aux sonorités plutôt nerveuses.

Harry eut la sensation qu'il allait peut-être pleurer.

«~Professeur Quirrell, si vous faites quoi que ce soit ressemblant à ce dont vous venez de parler, ça va me mettre en colère, et je ne voudrais vraiment pas me remettre en colère aujourd'hui…

--- Le but n'est \emph{pas} d'éviter de se mettre en colère,~» dit le professeur Quirrell, son visage très grave. «~La colère est naturelle. Vous devez apprendre à perdre même quand vous êtes en colère. Ou du moins à \emph{faire semblant} de perdre pour pouvoir ensuite \emph{planifier} votre vengeance. Comme je l'ai fait aujourd'hui avec M. Goyle, à moins bien sûr que l'un d'entre vous ne pense qu'il \emph{est} réellement meilleur…

--- Je ne le suis pas~!~» cria M. Goyle depuis son bureau, l'air un peu paniqué. «~Je sais que vous n'avez pas vraiment perdu~! S'il vous plaît, ne planifiez pas de vengeance~!~»

Harry se sentit malade. Le professeur Quirrell ne connaissait pas son mystérieux côté obscur.

«~Professeur, nous avons vraiment besoin d'en parler après les cours…

--- Nous le ferons, dit le professeur Quirrell sur le ton d'une promesse. Après que vous aurez appris à perdre.~» Son expression était sérieuse. «~Il va sans dire que je vais exclure tout ce qui pourrait vous blesser ou même provoquer une douleur importante. La douleur viendra de la difficulté de perdre, de ne pas vous défendre et de ne pas renchérir jusqu'à gagner.~»

La respiration de Harry était devenue une série de halètements courts et paniqués. Il était plus effrayé qu'il ne l'avait été lorsqu'il avait quitté la salle de Potions.

«~Professeur Quirrell, parvint-il à dire, je ne veux pas que vous vous fassiez renvoyer à cause de ça…

--- Ça n'arrivera pas, dit le professeur Quirrell, si \emph{vous} leur dites ensuite que c'était nécessaire. Et je vous fais confiance pour ça.~» Pendant un moment, la voix du professeur Quirrell devint très sèche. «~Croyez-moi, ils ont toléré pire dans leurs couloirs. Ce cas ne sera exceptionnel que parce qu'il a lieu dans une salle de classe.

--- Professeur Quirrell~», murmura Harry, mais sa voix était quand même relayée partout, «~vous croyez vraiment que si je ne fais pas ça, je pourrais faire du mal à quelqu'un~?

--- Oui, dit simplement le professeur Quirrell.

--- Alors, Harry fut pris de nausée, je le ferai.~»

Le professeur Quirrell pivota pour faire face aux Serpentard. «~Donc… avec l'approbation complète de votre enseignant, et afin que Rogue ne puisse être blâmé pour vos actes… l'un ou l'une d'entre vous veut-il prouver qu'il domine le Survivant~? Le pousser en tous sens, le projeter au sol, l'entendre implorer votre pitié~?~»

Cinq mains se levèrent.

«~Tous ceux qui ont levé la main, vous êtes des idiots finis. Quelle partie de \emph{faire semblant de perdre} n'avez-vous pas compris~? Si Harry Potter devient le prochain Seigneur des Ténèbres, il vous pourchassera et vous tuera après avoir obtenu son diplôme.~»

Les cinq mains retombèrent brutalement sur leurs pupitres.

«~Je ne le ferai pas~», dit Harry, sa voix assez faible. «~Je jure de ne jamais me venger sur ceux qui m'aideront à apprendre à perdre. Professeur Quirrell… pourriez-vous \emph{s'il vous plaît}… arrêter de faire ça~?~»

Le professeur Quirrell soupira.

«~Je \emph{suis} navré, M. Potter. Je me rends compte que vous devez trouver cela tout autant agaçant, que vous projetiez de devenir le prochain Seigneur des Ténèbres ou non. Mais ces enfants avaient \emph{aussi} une importante leçon de vie à apprendre. Accepteriez-vous que je vous octroie un point Quirrell en guise d'excuse~?

--- Disons deux~», dit Harry.

Il y eut un remous de rires surpris, ce qui désamorça une partie de la tension ambiante.

«~Fait, dit le professeur Quirrell.

---Et après avoir obtenu mon diplôme, je vous pourchasserai et je vous \emph{chatouillerai}.~»

Il y eut plus de rires, mais le professeur Quirrell ne sourit pas.

Harry avait l'impression de lutter contre un anaconda, d'essayer de forcer la conversation vers l'étroit chemin qui permettrait aux gens de se rendre compte qu'il n'était pas un Seigneur des Ténèbres après tout… \emph{pourquoi} le professeur Quirrell était-il si suspicieux à son égard~?

«~Professeur, dit la voix non amplifiée de Drago. Ce n'est pas non plus mon ambition que de devenir un Seigneur des Ténèbres stupide.~»

Il y eut un silence choqué dans la salle.

\emph{Tu n'as pas à faire ça~!} faillit lâcher Harry, mais il se maîtrisa à temps~; Drago ne souhaitait peut-être pas que l'on sache qu'il faisait cela par amitié pour Harry… ou par désir d'avoir l'air amical…

Appeler ça \emph{désir d'avoir l'air amical} donna à Harry la sensation qu'il était petit et méchant. Si Drago avait eut l'intention de l'impressionner, ça marchait à la perfection.

Le professeur Quirrell regardait Drago d'un air grave.

«~\emph{Vous} avez peur de ne pas pouvoir faire semblant de perdre, M. Malfoy~? Ce défaut qui décrit M. Potter vous décrit vous aussi~? Votre père vous a \emph{certainement} mieux éduqué que ça.

--- Lorsqu'il s'agit de parler, peut-être~», dit Drago, maintenant sur l'écran-relais. «~Pas lorsqu'il s'agit d'être poussé en tous sens et projeté au sol. Je veux être au moins aussi fort que vous, professeur Quirrell.~»

Les sourcils du professeur Quirrell s'élevèrent et restèrent levés. «~J'ai peur, M. Malfoy, dit-il après un moment, que les préparatifs que j'ai faits pour M. Potter, qui utilisent quelques Serpentard plus vieux à qui l'on dira \emph{plus tard} à quel point ils étaient stupides, ne marcheraient pas avec vous. Mais, selon mon opinion professionnelle, vous êtes déjà très fort. Si je devais apprendre que vous avez échoué, comme M. Potter a échoué aujourd'hui, je ferais les préparatifs appropriés et m'excuserais auprès de vous et de toute personne que vous auriez blessée. Je ne pense cependant pas que cela sera nécessaire.

--- Je comprends, professeur~», dit Drago.

Le professeur Quirrell regarda la classe. «~Quelqu'un d'autre souhaite-t-il devenir fort~?~»

Quelques étudiants regardèrent autour d'eux nerveusement. Certains, pensa Harry depuis le dernier rang, semblaient avoir ouvert leurs bouches, mais ils ne disaient rien. Personne ne se décida à parler.

«~Drago Malfoy sera l'un des généraux des armées de cette année, dit le professeur Quirrell, s'il daigne s'impliquer dans cette activité du soir. Et maintenant, M. Potter, merci de vous avancer.~»

\later

\emph{Oui}, avait dit le professeur Quirrell, \emph{ça doit être en face de tout le monde, en face de vos amis, parce que c'est là que Rogue s'est confronté à vous et c'est là que vous devez apprendre à perdre}.

Et maintenant les élèves de première année le regardaient. Dans un silence magiquement imposé, et priés par Harry et le professeur de ne pas intervenir. Hermione avait détourné son visage, mais elle n'avait rien dit ou même jeté un regard significatif, peut-être parce qu'elle avait été là en cours de Potions elle aussi.

Harry se tenait sur un tapis bleu rembourré, comme on aurait pu en trouver dans un dojo moldu, que le professeur Quirrell avait déposé au sol en prévision du moment où Harry serait projeté par terre.

Harry avait peur de ce qu'il risquait de faire. Si le professeur Quirrell avait raison au sujet de son intention de tuer…

La baguette de Harry gisait sur le bureau du professeur Quirrell, pas parce que Harry connaissait des sorts qui lui auraient permis de se défendre, mais parce que sinon (pensait Harry), il aurait pu essayer de la fourrer dans l'orbite de quelqu'un. Sa bourse gisait là aussi, contenant maintenant son Retourneur de Temps, toujours protégé mais toujours potentiellement fragile.

Harry avait plaidé auprès du professeur Quirrell pour qu'il lui transfigure des gants de boxe et les attache à ses mains. Le professeur Quirrell lui avait jeté un regard compréhensif et avait refusé.

\emph{Je ne viserai pas leurs yeux, je ne viserai pas leurs yeux, je ne viserai pas leurs yeux, ce serait la fin de ma vie à Poudlard, je serais arrêté}, se chanta Harry à lui-même, essayant de marteler la pensée dans son cerveau, espérant qu'elle resterait là si jamais son intention de tuer prenait le dessus.

Le professeur Quirrell revint. Il escortait treize Serpentard plus âgés, de différentes années. Harry reconnut l'un d'entre eux, celui qu'il avait frappé d'une tarte. Deux autres présents lors de cette confrontation étaient aussi là. Celui qui leur avait dit d'arrêter et qu'ils ne devraient vraiment pas faire ça n'était pas là.

«~Je répète, dit le professeur Quirrell d'une voix très sévère, Potter ne doit \emph{pas} être réellement blessé. Tout \emph{accident} sera considéré comme délibéré. Vous comprenez~?~»

Les autres Serpentard acquiescèrent en souriant.

«~Alors n'hésitez pas à remettre le Survivant à sa place~», dit le professeur Quirrell, avec un sourire tordu que seuls les élèves de première année comprirent.

Une sorte de consentement mutuel avait placé la cible-de-tarte au centre du groupe.

«~Potter, dit le professeur Quirrell, dites bonjour à M. Peregrine Derrick. Il est meilleur que vous et il est sur le point de vous le montrer.~»

Derrick s'avança et le cerveau de Harry hurla un cri distordu, il ne faut pas s'enfuir, il ne faut pas se défendre…

Derrick s'arrêta à une coudée de Harry.

Harry n'était pas encore en colère, juste effrayé. Parce qu'il faisait face à un jeune adolescent mâle plus grand que lui d'au moins cinquante centimètres, avec des muscles clairement définis, de la barbe, et un horrible sourire anticipatoire.

«~Demandez-lui de ne pas vous faire de mal, dit le professeur Quirrell. Peut-être que s'il vous trouve assez pathétique, il décidera que vous êtes ennuyeux et qu'il s'en ira.~»

Il y eut des rires venant des autres Serpentard plus âgés.

«~S'il te plaît~», dit Potter, sa voix vacillante, «~ne, me, fais pas, mal…

--- Ça n'avait pas l'air très sincère,~» dit le professeur Quirrell.

Le sourire de Derrick s'agrandit. L'imbécile maladroit avait un air très supérieur et…

… la température sanguine de Harry chutait…

«~S'il te plaît, ne me fais pas mal~», essaya à nouveau Harry.

Le professeur Quirrell secoua la tête. «~Par Merlin, comment êtes-vous parvenu à faire sonner ça comme une insulte, Potter~? Vous ne pouvez vous attendre qu'à une seule réponse de la part de M. Derrick.~»

Derrick s'avança délibérément et bouscula Harry.

Harry recula de quelques pieds, et, avant de pouvoir s'en empêcher, se raidit comme de la glace.

«~Faux, dit le professeur Quirrell, faux, faux, faux.

--- Tu m'as bousculé, Potter, dit Derrick. Excuse-toi.

--- Je suis désolé~!

--- Tu n'as pas l'\emph{air} désolé~», dit Derrick.

Les yeux de Harry s'écarquillèrent sous le coup de l'indignation, il \emph{était} parvenu à avoir l'air de plaider…

Derrick le poussa avec force, et Harry tomba sur le matelas sur ses mains et ses genoux.

Le tissu bleu semblait onduler non loin dans le champ de vision de Harry.

Il commençait à douter des réels motifs qui poussaient le professeur Quirrell à lui enseigner cette prétendue \emph{leçon}.

Un pied s'appuya sur l'arrière-train de Harry et un instant plus tard il était violemment poussé sur le côté, ce qui l'envoya s'étaler sur son dos.

Derrick rit. «~C'est \emph{vraiment amusant}~», dit-il.

Tout ce qu'il avait à faire était de dire que c'en était assez. Et de faire part de la chose au bureau du Directeur. Ce serait la fin du \emph{professeur de Défense} et de son infortuné passage à Poudlard et… le professeur McGonagall serait en colère, mais…

(Une image du professeur McGonagall lui apparut dans un flash, elle n'avait pas l'air en colère, seulement triste…)

«~Maintenant, dites-lui qu'il vaut mieux que vous, Potter, dit la voix du professeur Quirrell.

--- Tu vaux, mieux, que, moi.~»

Harry commença à se relever et Derrick lui mit un pied sur la poitrine et le repoussa sur le tapis.

Le monde devenait aussi transparent que du cristal. Les lignes d'actions et leurs conséquences s'étiraient devant Harry avec une clarté absolue. L'idiot ne s'attendrait pas à ce qu'il riposte, un rapide coup dans l'aine l'étourdirait assez longtemps pour…

«~Essayez encore~», dit le professeur Quirrell, et d'un mouvement soudain et très rapide, Harry fit une roulade et bondit sur ses pieds et virevolta face à l'endroit où se tenait son véritable ennemi, le professeur de Défense…

Le professeur Quirrell dit~: «~Vous n'avez aucune patience.~»

Harry vacilla. Son esprit, expert en pessimisme, dessina l'image d'un vieillard rabougri avec du sang s'écoulant de sa bouche après que Harry lui eut arraché la langue…

Un instant plus tard, Derrick poussa à nouveau Harry sur le tapis et s'assit sur lui, expulsant l'air de ses poumons.

«~Arrête~! hurla Harry. S'il te plaît, arrête~!

--- Mieux, dit le professeur Quirrell. Ça avait même l'air sincère.~»

Ça \emph{l'avait} été. C'était ça qui était horrible, qui le rendait malade~: ça \emph{avait} été sincère. Harry respirait par à-coups, la peur et la colère se répandaient en lui…

«~Perdez, dit le professeur Quirrell.

--- Je, perds, parvint à dire Harry.

---J'aime bien ça, dit Derrick de son perchoir. Perds encore un peu.~»

\later

Des mains poussaient Harry, l'envoyant trébucher d'un bout à l'autre du cercle de Serpentard plus âgés, jusqu'à une paire de mains qui le poussait à nouveau. Ça faisait longtemps que Harry avait arrêté d'essayer de ne pas pleurer, et il essayait maintenant juste de ne pas tomber.

«~Tu es un quoi, Potter~? dit Derrick.

--- Un, p-perdant, je perds, j'abandonne, tu gagnes, tu es m-meilleur, que moi, arrête s'il te plaît…

Harry trébucha sur un pied et alla s'écraser au sol, ses mains ne parvenant pas tout à fait à le rattraper. Il fut étourdi pendant un moment, puis il tenta de se remettre sur pied…

--- \emph{Assez~!}~» dit la voix du professeur Quirrell, et elle semblait assez tranchante pour couper de l'acier. «~Éloignez-vous de M. Potter~!~»

Harry vit l'air surpris sur leurs visages. Le frisson dans son sang qui avait monté puis décliné sourit avec une froide satisfaction.

Puis Harry s'effondra sur le tapis.

Le professeur Quirrell parla. Il y eut des bruits d'étranglement venant des Serpentard plus âgés.

«~Et je crois que l'héritier de Malfoy a quelque chose qu'il aimerait aussi vous expliquer~», conclut le professeur Quirrell.

La voix de Drago commença à parler. Elle semblait aussi tranchante que celle du professeur Quirrell, elle avait acquis la même cadence que celle que Drago utilisait pour imiter son père, et il disait des choses telles que \emph{aurait pu mettre en danger la Maison Serpentard} et \emph{qui sait combien d'alliés dans cette école} et \emph{absence totale de jugeote, sans parler de la ruse} et \emph{pauvres voyous}, \emph{seulement bons à faire des laquais} et quelque chose au fond du cerveau de Harry, en dépit de tout ce qu'il savait, désigna Drago comme étant un allié.

Harry avait mal partout, il était probablement contusionné, il avait froid, son esprit était complètement épuisé. Il essaya de penser à la chanson de Fumseck, mais sans la présence du phénix, il n'arrivait pas à se souvenir de la mélodie, et quand il essaya de l'imaginer, il ne sembla pas capable de penser à autre chose qu'à un gazouillis d'oiseau.

Puis Drago se tut et le professeur Quirrell dit aux Serpentard plus âgés qu'ils étaient congédiés, et Harry ouvrit les yeux et eut du mal à se mettre en posture assise,

«~Attendez~», dit Harry, forçant les mots à franchir ses lèvres, «~il y a quelque chose… que je veux, leur dire, à eux.

--- Attendez M. Potter~», dit froidement le professeur Quirrell aux Serpentard qui partaient.

Harry oscillait sur ses pieds. Il faisait attention à ne pas regarder en direction de ses camarades. Il ne voulait pas voir la façon dont ils le regardaient pour l'instant. Il ne voulait pas voir leur pitié.

Alors au lieu de ça, Harry regarda les Serpentard plus âgés, qui semblaient en état de choc. Ils le fixèrent en retour. Leurs visages étaient pleins d'effroi.

Son côté obscur, lorsqu'il avait repris le contrôle, s'était accroché à l'image de cet instant et avait continué à faire semblant de perdre.

Harry dit~:

«~Personne ne sera-
--- Arrêtez, dit le professeur Quirrell. Si c'est ce que je pense que c'est, merci d'attendre après leur départ. Ils l'apprendront plus tard. Nous avons tous nos leçons à apprendre, M. Potter.

--- Très bien, dit Harry.

--- Vous. Partez.~»

Les Serpentard plus âgés s'enfuirent et la porte se referma derrière eux.

«~Personne ne se vengera sur eux, dit Harry d'une voix rauque. C'est une requête envers quiconque se considère mon ami. J'avais une leçon à apprendre, et ils m'y ont aidé, ils avaient leur leçon à apprendre aussi, et c'est fini. Si vous racontez cette histoire, assurez-vous de raconter cette partie aussi.~»

Harry pivota pour regarder le professeur Quirrell.

«~Vous avez perdu~», dit le professeur Quirrell d'une voix qui pour la première fois était douce. C'était étrange venant du professeur, comme si sa voix n'aurait pas dû être capable de faire ça.

Harry \emph{avait} perdu. Il y avait eu des moments où la colère froide avait totalement disparu, remplacée par de la peur, et pendant ces moments, il avait supplié les Serpentard plus âgés, et il avait été sincère…

«~Et êtes-vous toujours en vie~?~» dit le professeur Quirrell, toujours avec cette étrange douceur.

Harry parvint à hocher la tête.

«~Toutes les défaites ne sont pas comme celle-là, dit le professeur Quirrell. Il y a des compromis et des capitulations négociées. Il existe d'autres moyens de calmer les petits durs. Il y a tout un art consistant à manipuler les autres en les laissant vous dominer. Mais d'abord, la défaite doit être \emph{envisageable}. Vous rappellerez-vous de la façon dont vous avez perdu~?

--- Oui.

--- Serez-vous capable de perdre~?

--- Je… pense…

--- Je pense aussi.~» Le professeur Quirrell s'inclina si bas que ses courts cheveux touchèrent presque le sol. «~Félicitations M. Potter, vous avez gagné.~»

Il n'y eut pas de source, pas de premier à agir, l'applaudissement démarra d'un coup comme un immense coup de tonnerre.

Harry ne pouvait pas dissimuler son air abasourdi. Il risqua un regard vers ses camarades et il vit leurs visages montrer non pas de la pitié mais de l'admiration. Les applaudissements venaient de Serdaigle et de Gryffondor et de Poufsouffle et même de Serpentard, probablement parce que Drago Malfoy applaudissait lui aussi. Certains étudiants se levaient de leurs chaises et la moitié de Gryffondor se tenait sur ses pupitres.

Alors Harry se tint là, chancelant, laissant le respect qu'ils éprouvaient pour lui le traverser, se sentant plus fort et peut-être même un peu soigné.

Le professeur Quirrell attendit que les applaudissements s'éteignent. Cela prit un bon moment.

«~Surpris, M. Potter~?~» dit le professeur Quirrell. Il avait l'air amusé. «~Vous venez de découvrir que le monde réel ne fonctionne pas \emph{toujours} comme dans vos pires cauchemars. Oui, si vous aviez été un pauvre petit garçon anonyme victime d'abus, alors ils vous auraient probablement respecté encore moins après ça, ils auraient eu pitié de vous tout en vous réconfortant depuis leur hautain perchoir. C'\emph{est} la nature humaine, j'en ai peur. Mais \emph{vous}, ils voyaient déjà en vous une figure de pouvoir. Et ils vous ont vu faire face à votre peur, et continuer à lui faire face, même si vous auriez pu partir n'importe quand. Avez-vous eu une moindre opinion de \emph{moi} lorsque j'ai délibérément enduré le fait d'être craché dessus~?~»

Harry sentit une sensation de brûlure dans sa gorge et la réprima frénétiquement. Il ne croyait pas assez à ce respect miraculeux pour se remettre à pleurer face à lui.

«~Votre réussite \emph{extraordinaire} à ce cours mérite une récompense extraordinaire, Harry Potter. Merci de l'accepter avec mes compliments, au nom de ma Maison, et souvenez-vous à partir d'aujourd'hui que tous les Serpentard ne sont pas semblables. Il y a des Serpentard, et il y a des Serpentard.~» Le professeur Quirrell eut un sourire plutôt large en disant cela. «~Cinquante-et-un points pour Serdaigle.~»

Il y eut une pause choquée et le tumulte commença alors chez les élèves de Serdaigle, des hurlements et des sifflements et des acclamations.

(Et au même instant Harry sentit qu'il y avait là quelque chose qui n'allait \emph{pas}, le professeur McGonagall avait eu raison, il y aurait \emph{dû} y avoir des conséquences, il y aurait dû y avoir un prix à payer, on ne pouvait pas juste tout remettre en place comme ça…)

Mais Harry vit l'exultation sur le visage des Serdaigle et il sut qu'il ne pouvait pas dire non.

Son cerveau fit une suggestion. C'était une bonne suggestion. Harry ne pouvait même pas croire que son cerveau le maintenait encore debout, et encore moins qu'il pouvait produire de bonnes suggestions.

«~Professeur Quirrell~», dit Harry, aussi clairement qu'il le pouvait à travers sa gorge brûlante. «~Vous êtes tout ce qu'un membre de votre Maison devrait être, et je pense que vous devez être exactement ce que Salazar Serpentard avait à l'esprit lorsqu'il participa à la fondation de Poudlard. Je vous remercie, vous et votre Maison,~» Drago hochait la tête très doucement et tournait légèrement l'un de ses doigts, \emph{continue}, «~et je pense que ça mérite trois acclamations pour Serpentard. Avec moi, tout le monde~?~» Harry marqua une pause. «~\emph{Huzzah}~!~». Seuls certains parvinrent à se joindre au premier essai. «~\emph{Huzzah}~!~» cette fois la plus grande partie de Serdaigle participa. «~\emph{Huzzah}~!~» C'était presque tout Serdaigle, quelques Poufsouffle éparpillés, et près d'un quart de Gryffondor.

La main de Drago leva son pouce d'un petit geste rapide.

La plupart des Serpentard affichaient des expressions de choc absolu. Quelques-uns fixaient le professeur Quirrell avec émerveillement. Blaise Zabini regardait Harry avec une expression intriguée et calculatrice.

Le professeur Quirrell s'inclina.

«~Merci à \emph{vous}, Harry Potter~», dit-il, souriant toujours de ce large sourire. Il se tourna vers la classe. «~Et maintenant, croyez-le ou non, nous avons encore une demi-heure avant la fin du cours, et c'est assez pour présenter le Bouclier Simple. M. Potter, bien sûr, va s'en aller et profiter d'un repos bien mérité.

--- Je peux…

--- Idiot~», dit affectueusement le professeur Quirrell. La classe riait déjà. «~Vos camarades pourront vous l'apprendre plus tard, ou je vous donnerai des cours particuliers si nécessaire. Mais \emph{maintenant}, vous allez prendre la troisième porte en partant de la gauche à l'arrière de cette estrade, où vous trouverez un lit, un assortiment de casse-croûtes exceptionnellement délicieux, et quelques lectures extrêmement légères tirées de la bibliothèque de Poudlard. Vous ne pouvez rien emmener d'autre, et particulièrement pas vos manuels. Maintenant partez.~»

Harry s'en fut.
%  LocalWords:  raco Mornelithe Falconsbane

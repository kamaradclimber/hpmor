\chapter{Hypothèse de l'intelligence machiavélique}

\iftoggle{embedepigraphs}{J. K. Rowling coils and strikes, unseen; Orca circles, hard and lean.

}{}

\section{Acte 3~:}

\lettrine{D}{rago} attendait, l'estomac noué, dans une petite alcôve munie d'une fenêtre qu'il avait trouvée près de la Grande Salle.

Il y aurait un prix à payer, et il serait élevé. Drago l'avait su dès qu'il s'était réveillé et qu'il s'était rendu compte qu'il n'osait pas entrer dans la Grande Salle pour le petit déjeuner, de peur d'y voir Harry Potter car il ne savait pas ce qui se passerait ensuite.

Des bruits de pas approchaient.

«~Eh ben le v'la, dit la voix de Vincent, mais l'boss est pas d'bonne humeur aujourd'hui, alors fais gaffe où tu mets les pieds.~»

Drago allait le dépecer vivant et renvoyer le corps de cet idiot accompagné d'une requête pour un serviteur plus intelligent comme s'il n'avait été qu'une gerbille morte.

Un des bruits de pas s'éloigna et l'autre se fit plus proche.

La sensation dans l'estomac de Drago devint encore pire.

Harry Potter entra dans son champ de vision. Son expression faciale était précautionneusement maîtrisée, mais sa robe à ourlet bleu semblait étrangement de travers, comme si elle n'avait pas été remise correctement…

«~\emph{Ta main}~», dit Drago sans réfléchir.

Harry leva son bras gauche comme pour l'inspecter lui-même.

La main pendait mollement, l'air morte.

«~Madame Pomfresh a dit que ce n'était pas permanent, dit doucement Harry, et que j'aurais en grande partie récupéré demain, quand les cours commenceront.~»

Pendant un instant, la nouvelle lui procura un soulagement.

Et il comprit alors.

«~Tu es allé voir Madame Pomfresh, chuchota Drago.

--- Bien sûr~», dit Harry Potter, comme s'il énonçait l'évidence. «~Ma main ne fonctionnait pas.~»

Drago comprit lentement à quel point il avait été un imbécile \emph{complet}, bien pire que les Serpentard plus âgés qu'il avait incendiés plus tôt.

Il avait simplement tenu pour acquis que personne n'irait voir les autorités pour se plaindre d'un Malfoy. Que personne ne voudrait attirer l'attention de Lucius. Jamais.

Mais Harry Potter n'était pas un petit Poufsouffle effrayé essayant de rester hors du jeu. Il jouait déjà au jeu, et il avait déjà toute l'attention de Père.

«~Qu'a-t-elle dit d'autre~?~», dit Drago, le cœur au bord des lèvres.

«~Le professeur Flitwick a dit que le sort qui a été jeté sur ma main est un sombre maléfice de torture et que c'est une affaire très sérieuse et qu'il est inacceptable que je refuse de dire qui avait fait ça.~»

Il y eut une longue pause.

«~Et après~?~» dit Drago d'une voix tremblante.

Harry Potter sourit légèrement. «~Je me suis confondu en excuses, ce qui a donné au professeur Flitwick un air \emph{très} sévère, et alors je lui ai dit que toute cette affaire était en effet extrêmement sérieuse, secrète et \emph{délicate}, et que j'avais déjà informé le directeur de ce projet.~»

Drago manqua d'air. «~Non~! Flitwick ne va pas simplement accepter ça~! Il ira vérifier auprès de Dumbledore~!

--- En effet, dit Harry Potter. J'ai été promptement transbahuté jusqu'au bureau du directeur.~»

Drago tremblait à présent. Si Dumbledore amenait Harry Potter devant le Magenmagot, de son plein gré ou pas, et faisait témoigner le Survivant, sous l'influence du Veritaserum, du fait que Drago l'avait torturé… trop de gens aimaient Harry Potter, Père pourrait \emph{perdre} ce vote…

Père pourrait convaincre Dumbledore de ne pas faire ça, mais il lui en \emph{coûterait}. Un coût terrible. Le jeu avait maintenant des règles, on ne pouvait plus menacer les gens au hasard. Mais Drago s'était jeté entre les mains de Dumbledore de son plein gré. Et Drago était un otage de grande valeur.

Même si maintenant que Drago ne pouvait plus devenir un Mangemort, il en avait moins que Père ne le pensait.

La pensée déchira le cœur de Drago comme l'aurait fait un sortilège de Coupure.

«~Et alors~? murmura Drago.

--- Dumbledore a immédiatement déduit que c'était toi. Il sait qu'on se fréquente.~»

Le pire des scénarios possibles. Si Dumbledore n'avait pas deviné qui était le responsable, il n'aurait peut-être pas pris le risque d'utiliser la Legilimancie juste pour le découvrir… mais s'il le \emph{savait}…

«~Et~?~» se força à prononcer Drago.

«~Nous avons eu une petite conversation.

--- Et~?~»

Harry sourit. «~Et je lui ai expliqué qu'il serait dans son intérêt de ne rien faire du tout.~»

L'esprit de Drago arriva à pleine allure sur un mur de briques et explosa. Il se contenta de fixer Harry Potter, la bouche mollement ouverte, comme s'il avait été complètement stupide.

Il fallut longtemps avant que Drago ne se souvienne.

Harry connaissait le mystérieux secret de Dumbledore, celui que Rogue utilisait pour maintenir son emprise.

Drago pouvait maintenant se l'imaginer. Dumbledore, l'air sévère, dissimulant son impatience tandis qu'il expliquait à Harry à quel point toute cette affaire était effroyablement sérieuse.

Et Harry lui disant poliment de la fermer s'il tenait à son secret.

Père avait mis Drago en garde contre ce genre de personnes, les gens qui pouvaient vous mener à votre ruine mais être si aimables que c'en était difficile de les haïr correctement.

«~Après quoi, dit Harry, le directeur a dit au professeur Flitwick que c'était bel et bien une affaire secrète et délicate dont il avait déjà été informé et qu'il ne pensait pas qu'en rajouter serait pour l'instant d'une grande aide, ni pour moi ni pour qui que ce soit. Le professeur Flitwick a commencé à dire quelque chose concernant le fait que les intrigues de Dumbledore allaient bien trop loin, et j'ai alors dû l'interrompre pour lui expliquer que ça avait été mon idée et pas quelque chose que le directeur m'avait forcé à faire, alors le professeur Flitwick s'est tourné vers moi et a commencé à \emph{me} faire la leçon, et le directeur l'a interrompu \emph{lui},et il a dit qu'étant le Survivant j'étais destiné à avoir des aventures bizarres et dangereuses et qu'il était plus sûr que je fonce dedans plutôt que j'attende qu'elles m'arrivent par accident, et c'est là que le professeur Flitwick a levé ses petites mains au ciel et a commencé à nous crier dessus \emph{à tous les deux} d'une voix haut perchée en disant qu'il se fichait bien de ce qu'on pouvait être en train de mijoter, mais que ça ne devrait plus jamais se produire tant que j'étais Serdaigle ou que sinon il me ferait renvoyer et que je pourrais aller à Gryffondor où toutes ces \emph{Dumbledories} avaient leur place…~»

Harry rendait les choses \emph{très} difficiles pour Drago. Comment allait-il le haïr maintenant~?

«~Bref, dit Harry, je ne voulais pas être renvoyé de Serdaigle, alors j'ai promis au professeur Flitwick que rien de tel n'aurait plus jamais lieu, et que si ça se reproduisait, je lui dirais qui était responsable.~»

Les yeux de Harry auraient dû être froids. Ce n'était pas le cas. Sa voix aurait dû laisser planer une menace mortelle. Elle ne le fit pas.

Et Drago vit la question qui aurait dû lui paraître évidente, et ça cassa instantanément l'ambiance.

«~Pourquoi… n'as-tu pas…~»

Harry marcha jusqu'au petit rayon de soleil qui brillait dans l'alcôve et tourna la tête vers l'extérieur, vers les verts terrains de Poudlard. La lumière l'enveloppait, éclairait sa robe, illuminait son visage.

«~Pourquoi~?~» dit Harry. Sa voix s'interrompit. «~J'imagine que c'est parce que je ne pouvais juste pas me sentir en colère contre toi. Je savais que c'était moi qui t'avais d'abord fait du mal. Je ne dirais même pas qu'on est quitte, parce que ce que je t'ai fait est pire que ce que tu m'as fait.~»

C'était comme de foncer dans un autre mur de briques. Harry aurait aussi bien pu parler en grec ancien vu ce que Drago y comprenait.

L'esprit de Drago partit désespérément à la recherche de motifs connus et revint parfaitement bredouille. Cette dernière phrase avait été une concession qui n'était pas dans l'intérêt de Harry. Ce n'était même pas ce que Harry aurait dû dire pour faire de Drago un loyal serviteur, maintenant qu'il avait du pouvoir sur lui. Pour ça, Harry aurait dû mettre l'emphase sur son immense gentillesse, pas sur le fait qu'il avait fait du mal à Drago.

«~Mais quand même~», dit Harry, et sa voix était maintenant plus basse, presque un murmure, «~Drago, ne recommence pas s'il te plaît. Ça m'a fait mal, et je ne suis pas sûr que je pourrais te pardonner une seconde fois. Je ne sais pas si je pourrais le vouloir.~»

Drago ne comprenait pas.

Harry essayait-il d'être son \emph{ami}~?

Il était impensable que Harry Potter soit assez stupide pour croire que c'était encore possible après ce qu'il avait fait.

On pouvait être l'allié et l'ami de quelqu'un, comme Drago avait essayé avec Harry, ou on pouvait détruire sa vie et ne lui laisser aucune option. On ne pouvait pas faire les deux.

Mais alors Drago ne comprenait pas ce que Harry \emph{pouvait bien} être en train d'essayer de faire.

Et une étrange idée lui vint, quelque chose dont Harry n'avait pas arrêté de parler hier.

Et la pensée était~: \emph{Fais un test}.

\emph{Tu es éveillé à la science maintenant}, avait dit Harry, \emph{et même si tu n'apprends jamais à utiliser ton pouvoir, tu seras… toujours… à la recherche… de moyens… de tester… tes croyances…} les mots prophétiques, prononcés entre des halètements d'agonie, n'avaient pas cessé de tournoyer dans l'esprit de Drago.

Si Harry \emph{faisait} semblant d'être l'ami repentant qui avait accidentellement fait souffrir quelqu'un…

«~Tu avais \emph{prévu de faire} ce que tu m'as fait~!~» dit Drago, parvenant à glisser une note accusatrice dans sa voix. «~Tu ne l'as pas fait parce que tu t'es mis en colère, tu l'as fait parce que tu \emph{voulais} le faire~!~»

\emph{Idiot}, dirait alors Harry Potter, \emph{bien sûr que je l'ai prévu, et maintenant tu es à moi…}

Harry se retourna vers Drago.

«~Ce qui s'est passé hier ne faisait \emph{pas} partie du plan~», dit Harry, et sa voix semblait être coincée dans sa gorge. «~Le \emph{plan}, c'était que je t'enseigne pourquoi ce serait toujours mieux pour toi de connaître la vérité, et alors nous aurions essayé de découvrir la vérité au sujet du sang, et quelle qu'ait été la réponse nous l'aurions acceptée. Hier j'ai… précipité les choses.

--- Toujours mieux pour moi de connaître la vérité,~» dit froidement Drago. «~Comme si tu m'avais fait une \emph{faveur}.~»

Harry hocha la tête -- ce qui laissa Drago complètement scié -- puis il dit~: «~Et si Lucius a la même idée que j'ai eue, que le problème est que les sorciers les plus forts n'ont pas assez d'enfants~? Il pourrait mettre en place un plan de rétribution des sang-pur les plus forts afin de les faire se reproduire. En fait, si le purisme du sang \emph{était} vrai, c'est exactement ce que Lucius \emph{devrait} faire -- affronter le problème de \emph{son} côté, là où il peut provoquer des changements tout de suite. Drago, tu es pour l'instant le seul ami de Lucius qui serait capable de l'empêcher de faire des efforts inutiles, parce que tu es le seul qui connaît la \emph{vraie} vérité et qui peut prédire les vrais résultats.~»

L'idée vint à Drago que Harry Potter avait été élevé dans un lieu tellement étrange qu'il tenait plus de la créature magique que du sorcier. Drago ne pouvait tout simplement pas deviner ce que Harry allait maintenant dire ou faire.

«~\emph{Pourquoi}~?~» dit Drago. Instiller de la douleur et un sentiment de trahison dans sa voix n'était pas difficile du tout. «~Pourquoi m'as-tu \emph{fait} ça~? Quel \emph{était} ton plan~?

--- Eh bien, dit Harry, tu es l'héritier de Lucius, et crois-le ou non, Dumbledore croit que je lui appartiens. Nous pourrions donc grandir et nous affronter. Ou nous pourrions faire autre chose.~»

Lentement, l'esprit de Drago absorba cette idée. «~Tu veux provoquer un duel à mort entre eux, puis récupérer le pouvoir après qu'ils se seront tous les deux épuisés.~» Drago sentit la terreur monter dans sa poitrine. Il \emph{fallait} qu'il arrête ce plan, peu importe ce que ça lui coûterait…

Mais Harry secoua la tête.

«~Par les cieux, \emph{non~!}

--- Non…~?

--- Tu ne serais pas partant, et moi non plus, dit Harry. C'est \emph{notre} monde, on ne veut pas le briser. Mais, par exemple, imagine que Lucius pense que la Conspiration est ton outil et que tu es de son côté, que Dumbledore pense que la Conspiration est mienne et que je suis de son côté, que Lucius pense que tu m'as fait changer de camp et que Dumbledore croit que la Conspiration est mienne, et que Dumbledore pense que je t'ai fait changer de camp et que Lucius croit que la Conspiration est tienne. Ils nous aideraient alors tous les deux mais seulement de façon à ce que l'autre ne puisse le remarquer.~»

Drago n'eut pas à faire semblant d'être sans voix.

Père l'avait un jour emmené voir une pièce nommée \emph{La Tragédie de Light}, qui racontait l'histoire de ce Serpentard \emph{incroyablement intelligent} nommé Light qui s'était résolu à purifier le monde de toute sa noirceur en utilisant un anneau ancien capable de tuer toute personne dont il aurait connu le nom et le visage, et ce Light faisait face à un autre Serpentard incroyablement intelligent, un méchant nommé Lawliet, qui portait un déguisement lui permettant de dissimuler son vrai visage~; Drago avait crié et s'était réjoui à tous les bons moments, en particulier au milieu~; et alors la pièce avait mal fini et Drago avait été énormément déçu et Père avait gentiment fait remarquer que le mot Tragédie était là juste au milieu du titre.

Ensuite, Père avait demandé à Drago s'il comprenait pourquoi ils étaient allés voir cette pièce.

Drago avait répondu que c'était pour lui apprendre à être aussi fourbe que Light et Lawliet quand il serait grand.

Père avait dit à Drago qu'il aurait difficilement pu plus se tromper et il lui avait fait remarquer que, bien que Lawliet ait intelligemment dissimulé son visage, il n'avait eu aucune bonne raison de donner son \emph{nom} à Light. Père avait alors continué de démolir à peu près chaque partie de la pièce et Drago avait écouté, ses yeux s'écarquillant de plus en plus. Et Père avait fini par dire que les pièces comme celle-ci étaient \emph{toujours} invraisemblables, parce que si le dramaturge avait su ce que quelqu'un de \emph{vraiment} aussi intelligent que Light aurait \emph{vraiment} fait, il aurait essayé de conquérir le monde lui-même au lieu de juste écrire des pièces sur le sujet.

C'est alors que Père avait enseigné à Drago la Règle des Trois, qui disait qu'un complot nécessitant que trois choses différentes surviennent ne fonctionnerait jamais dans le monde réel.

Père avait \emph{ensuite} expliqué que puisque seul un idiot essaierait de faire un complot \emph{aussi compliqué que possible}, la vraie limite était deux.

Drago ne pouvait pas trouver les mots qui lui auraient permis de décrire la gargantuesque infaisabilité du plan de Harry.

Mais c'était \emph{exactement} le genre d'erreur que vous faisiez quand vous n'aviez pas de mentors et que vous pensiez que vous étiez intelligents et que vous aviez appris à comploter en regardant des pièces de théâtre.

«~Donc, dit Harry, que penses-tu de ce plan~?

--- C'est malin…~» dit lentement Drago. S'écrier \emph{génial}~! et manquer d'air sous l'effet de l'admiration aurait été trop suspicieux. «~Harry, je peux te poser une question~?

--- Bien sûr, dit Harry.

--- Pourquoi as-tu acheté une bourse coûteuse à Granger~?

--- Pour ne pas avoir l'air rancunier, répondit immédiatement Harry. Mais je m'attends aussi à ce qu'elle éprouve de la difficulté à refuser toute petite requête que je pourrais lui faire dans les deux mois à venir.~»

Et c'est alors que Drago comprit que Harry essayait \emph{vraiment} d'être son ami.

Le coup de Harry contre Granger \emph{avait été} intelligent, peut-être même brillant. Diminuer les soupçons de votre ennemi \emph{et} en faire votre débiteur grâce à un moyen amical, afin de pouvoir leur faire faire ce que vous vouliez qu'ils fassent \emph{simplement en le leur demandant}. Drago n'aurait pas pu faire ça car sa cible aurait eu trop de soupçons, mais le Survivant, lui, \emph{pouvait}. La première étape du plan de Harry était donc d'offrir un cadeau très cher à son ennemi. Drago n'y aurait pas pensé mais ça pouvait \emph{marcher}…

Si vous étiez l'ennemi de Harry, ses plans seraient peut-être difficiles à déchiffrer au premier abord, ils auraient peut-être même l'air stupides, mais une fois que vous les auriez compris, son raisonnement se \emph{tiendrait}, vous comprendriez qu'il essayait de vous faire du mal.

La façon dont Harry se comportait envers Drago n'avait \emph{aucun} sens.

Parce que si vous étiez l'\emph{ami} de Harry, alors il essayait d'être votre ami de la façon étrange et incompréhensible que les Moldus lui avaient enseignée, même si ça voulait dire qu'il allait détruire votre vie.

Le silence s'étira.

«~Je sais que j'ai horriblement malmené notre amitié, dit enfin Harry. Mais Drago, s'il te plaît, rends-toi compte que je voulais juste que nous découvrions la vérité ensemble. Est-ce quelque chose que tu peux me pardonner~?~»

Deux sentiers qui bifurquent, mais un seul où le retour en arrière serait facile si jamais Drago changeait d'avis…

«~Je crois que je comprends ce que tu essayais de faire, mentit Drago, donc oui.~»

Les yeux de Harry s'éclairèrent. «~Je suis heureux d'entendre ça, Drago~», dit-il doucement.

Les deux élèves se tenaient dans l'alcôve, Harry trempé dans le rayon de soleil solitaire, Drago dans la pénombre.

Et Drago se rendit compte avec horreur et désespoir que, bien que ce fut un funeste destin que d'être l'ami de Harry, ce dernier avait maintenant tellement de moyens de le menacer qu'être son ennemi serait encore \emph{pire}.

Probablement.

Peut-être.

Eh bien, il pourrait toujours changer et devenir son ennemi plus tard…

Il était foutu.

«~Et donc, dit Drago, on fait quoi maintenant~?

--- On étudie encore samedi prochain~?

--- Ça a intérêt à ne pas être comme la dernière fois…

--- Ne t'en fais pas, ce ne sera pas le cas, dit Harry. Quelques samedis de plus comme \emph{celui-ci} et tu \emph{me} dépasseras.~»

Harry rit. Pas Drago.

«~Oh, et avant que tu y ailles~», dit Harry, et il sourit d'un air penaud. «~Je sais que ce n'est pas un moment bien choisi, mais à vrai dire je voulais te demander conseil.

--- D'accord~», dit Drago, encore distrait par la phrase précédente.

Le regard de Harry s'intensifia. «~Acheter cette bourse pour Granger a utilisé la majeure partie de l'argent que j'étais parvenu à voler dans mon coffre-fort de Gringotts…~»

Quoi.

«~… et McGonagall a la clé du coffre-fort, ou bien peut-être que c'est Dumbledore qui l'a maintenant. Et j'allais juste commencer un plan qui va coûter pas mal d'argent, alors je me demandais si tu savais comment je pourrais accéder…

--- Je te prêterai l'argent~», dit la bouche de Drago, mue par un pur réflexe existentiel.

Harry sembla surpris -- agréablement.

«~Drago, tu n'as pas à…

--- Combien~?~»

Harry énonça le montant et Drago ne put empêcher le choc d'être visible sur son visage. C'était à peu près tout l'argent de poche que Père lui avait donné pour l'année, Drago n'aurait plus que quelques Gallions…

Puis Drago se frappa mentalement. Tout ce qu'il avait à faire était d'écrire à Père que l'argent avait disparu parce qu'il était parvenu à le \emph{prêter à Harry Potter}, et Père lui enverrait un message de félicitations écrit à l'encre d'or, une grenouille en chocolat géante qu'il mettrait deux semaines à manger et le décuple de la somme juste au cas où Harry Potter aurait besoin d'un autre prêt.

«~C'est beaucoup trop, c'est ça, dit Harry. Je suis désolé, je n'aurais pas dû te demander…

--- Excuse-moi mais tu sais, je \emph{suis} un Malfoy, dit Drago. J'étais juste surpris que tu en \emph{veuilles} autant.

--- Ne t'en fais pas, dit Harry Potter d'un ton réjoui. Ce n'est rien qui menace les intérêts de ta famille, c'est juste moi qui m'amuse à être malfaisant.~»

Drago hocha la tête.

«~Pas de problème alors. Tu les veux tout de suite~?

--- Avec plaisir, dit Harry.

Tandis qu'ils quittaient l'alcôve et commençaient à se diriger vers les donjons, Drago ne put s'empêcher de demander~:

«~Alors \emph{peux}-tu me dire contre qui tu fomentes~?

--- Rita Skeeter.~»

Drago pensa quelques très gros mots à sa propre intention, mais il était trop tard pour dire non.

\later

Drago avait déjà commencé à retrouver ses esprits lorsqu'ils arrivèrent aux donjons.

Il \emph{avait} du mal à haïr Harry Potter. Harry \emph{avait} essayé d'être amical, c'est juste qu'il était dingue.

Et ça n'allait pas arrêter la vengeance de Drago, ni même la ralentir.

«~Donc~», dit Drago après avoir observé les alentours pour s'assurer que personne n'était proche. Leur voix était bien sûr brouillée, mais ça ne pouvait pas faire de mal d'être parfaitement certain. «~J'ai réfléchi. En amenant de nouvelles recrues dans la Conspiration, il faudra qu'ils \emph{croient} être nos égaux. Sinon, il suffirait de l'\emph{un} d'entre eux pour que le complot soit éventé auprès de Père. Tu as déjà compris ça, bien sûr~?

--- Naturellement, dit Harry.

--- \emph{Serons-}nous égaux~?~» dit Drago.

---J'ai bien peur que non~», dit Harry. Il était clair qu'il essayait d'être gentil, et il était tout aussi clair qu'il essayait d'effacer une bonne dose de condescendance de sa voix et qu'il n'y arrivait pas tout à fait. «~Je suis désolé Drago, mais pour l'instant tu ne sais même pas ce que le mot \emph{Bayésien} dans \emph{Conspiration} \emph{Bayésienne} veut dire. Il va falloir que tu étudies pendant des mois avant que nous puissions enrôler quelqu'un d'autre, juste pour que tu puisses faire \emph{bonne mesure}.

--- Parce que je ne connais pas assez de science~», dit Drago, gardant sa voix précautionneusement neutre.

En entendant ça, Harry secoua sa tête.

«~Le problème n'est pas que tu ignores des faits scientifiques précis tels que l'acide désoxyribonucléique. \emph{Cela} ne t'empêcherait pas d'être mon égal. Le problème est que tu n'as pas été entraîné aux méthodes de la rationalité, aux secrets \emph{plus profonds} qui ont permis à toutes ces découvertes d'être faites. J'\emph{essaierai} de te les enseigner, mais ils sont beaucoup plus difficiles à apprendre. Pense à ce que nous avons fait aujourd'hui, Drago. Oui, tu as fait une partie du travail. Mais c'était moi qui tenais la barre. Tu as répondu à certaines des questions. Je les ai toutes posées. Tu as donné de la force. J'ai tenu le gouvernail tout seul. Et sans les méthodes de la rationalité, Drago, il est impossible que tu mènes la Conspiration là où elle doit aller.

--- Je vois,~» dit Drago, l'air déçu.

La voix de Harry essaya de devenir encore plus gentille.

«~Drago, j'essaierai de respecter ton expertise dans les trucs politiques. Mais tu dois aussi respecter mon expertise, et il est tout simplement \emph{impossible} que tu sois mon égal lorsqu'il s'agit de manœuvrer la Conspiration. Tu es un scientifique depuis \emph{hier}, tu connais \emph{un} secret au sujet de l'acide désoxyribonucléique et tu n'as reçu d'entraînement dans \emph{aucune} des méthodes de la rationalité.

---Je comprends~», dit Drago.

Et il comprenait.

\emph{Trucs politiques}, avait dit Harry. Prendre contrôle de la Conspiration ne serait probablement même pas difficile. Et après, il tuerait Harry juste pour être sûr…

Le souvenir de la façon dont il s'était senti malade la nuit dernière lui revint, quand il avait su que Harry hurlait.

Drago pensa d'autres gros mots.

Très bien. Il ne \emph{tuerait pas} Harry. Harry avait été élevé par des Moldus, ce n'était pas sa faute s'il était dingue.

Au lieu de ça, Harry survivrait, juste pour que Drago puisse lui dire que tout cela avait été pour le bien de Harry, vraiment, il devrait le remercier…

Et dans une convulsion de plaisir surpris, Drago se rendit compte que c'\emph{était} vraiment pour le bien de Harry. Si Harry essayait de prendre Dumbledore et Père pour des idiots, il \emph{mourrait}.

Voilà qui était \emph{parfait}.

Drago prendrait tous ses rêves à Harry, comme Harry lui avait pris les siens.

Drago dirait à Harry que ça avait été pour son propre bien, et ce serait la pure vérité.

Drago manierait la Conspiration et le pouvoir de la science pour purifier le monde magique, et Père serait aussi fier de lui que s'il avait été un Mangemort.

Les plans maléfiques de Harry seraient déjoués, et les forces du bien prévaudraient.

La vengeance parfaite.

À moins que…

\emph{Fais juste semblant de faire semblant d'être un scientifique}, lui avait dit Harry.

Drago ne possédait pas le vocabulaire qui lui aurait permis de décrire précisément ce qui clochait dans l'esprit de Harry…

(puisque Drago n'avait jamais entendu le terme \emph{profondeur de récursion})

… mais il pouvait deviner le genre de plans qui en jailliraient.

… à moins que tout cela ne soit exactement ce que Harry \emph{voulait} que Drago fasse, que cela faisait partie intégrante d'un autre plan \emph{plus large encore} dans lequel Drago allait \emph{tomber} en essayant de déjouer celui-ci, Harry \emph{savait} même peut-être que son plan était impossible, il n'avait peut-être pas d'autre but que de \emph{provoquer} Drago à essayer de le contrecarrer…

Non. Au bout de cette route se trouvait \emph{la folie}. Il \emph{fallait} qu'il y ait une limite. Le Seigneur des Ténèbres lui-même n'avait pas été si retors. Ce genre de chose n'arrivait pas dans la vraie vie, seulement dans les histoires idiotes que Père lui racontait au sujet de stupides gargouilles qui finissaient toujours par aider les plans du héros à chaque fois qu'elles essayaient de l'arrêter.

\later

Et Harry marchait aux côtés de Drago, un sourire sur le visage, en réfléchissant aux origines évolutives de l'intelligence humaine.

Au début, avant que les gens aient vraiment compris comment l'évolution fonctionnait, ils étaient allés pêcher des idées saugrenues telles que \emph{l'intelligence humaine a évolué pour qu'on puisse inventer de meilleurs outils}.

La raison pour laquelle c'était saugrenu, c'était qu'il suffisait qu'une seule personne de la tribu invente un outil pour que tout le monde puisse l'utiliser, et il circulerait jusqu'aux autres tribus, et il serait toujours utilisé par ses descendants cent ans plus tard. C'était génial du point de vue du progrès scientifique, mais en termes évolutionnistes, ça voulait dire que la personne qui avait inventé quelque chose n'avait pas un très grand \emph{avantage} sélectif, qu'elle n'avait pas \emph{beaucoup} plus d'enfants que les autres. Seuls les avantages sélectifs \emph{relatifs} pouvaient augmenter la fréquence relative d'un gène dans une population et mener une mutation isolée à un stade où elle serait universelle et que tout le monde l'aurait. Et les inventions géniales n'étaient tout simplement pas assez fréquentes pour fournir le genre de pression sélective nécessaire à la promotion d'une mutation. Si vous regardiez les humains avec leurs pistolets et leurs tanks et leurs armes nucléaires et que vous les compariez aux chimpanzés, il était normal de faire la conjecture que l'intelligence était là pour produire de la technologie. Une conjecture normale mais fausse.

Avant que les gens aient vraiment compris comment l'évolution fonctionnait, ils étaient allés pêcher des idées saugrenues telles que \emph{le climat a changé, les tribus ont dû migrer, et les gens ont dû devenir plus intelligents afin de pouvoir résoudre tous les nouveaux problèmes}.

Mais les humains ont des cerveaux quatre fois plus grands que ceux d'un chimpanzé. 20~\% de l'énergie métabolique d'un humain sert à nourrir son cerveau. Le point auquel les humains étaient plus intelligents que toutes les autres espèces atteignait un niveau \emph{absurde}. Ce genre de chose n'était pas arrivé parce que l'environnement avait légèrement augmenté la difficulté des problèmes. Les organismes seraient alors devenus juste un peu plus intelligents afin de pouvoir les résoudre. Se retrouver avec cet énorme cerveau surdimensionné avait dû être le fruit d'un processus évolutif \emph{débridé}, quelque chose qui avait poussé et poussé sans aucune limite.

Et les scientifiques d'aujourd'hui avaient une assez bonne conjecture quant à ce que ce processus évolutif débridé avait été.

Harry avait un jour lu un célèbre livre intitulé \emph{La politique du chimpanzé}. Le livre décrivait la façon dont un chimpanzé adulte prénommé Luit avait confronté l'alpha vieillissant, Yeroen, avec l'aide d'un jeune chimpanzé récemment devenu adulte prénommé Nikkie. Nikkie n'était pas intervenu directement dans les combats entre Luit et Yeroen, mais il avait empêché les partisans de Yeroen de lui venir directement en aide en les distrayant à chaque fois qu'une confrontation entre Luit et Yeroen survenait. Et Luit avait fini par gagner, et il était devenu le nouvel alpha, avec Nikkie pour second…

… mais il n'avait pas fallu longtemps avant que Nikkie ne forme une alliance avec Yeroen, ne renverse Luit, et ne devienne le \emph{nouveau} nouvel alpha.

Ça permettait vraiment d'apprécier ce à quoi des millions d'années d'hominidés essayant d'être plus malins \emph{que les autres} -- une course à l'armement évolutionniste sans limites -- avaient menés en termes de capacité intellectuelle.

Pasque, voyez, un humain l'aurait carrément vu v'nir.

\later

Et Drago marchait aux côtés de Harry, contenant son sourire tandis qu'il songeait à sa vengeance.

Un jour, peut-être dans plusieurs années, mais un jour, Harry Potter apprendrait exactement ce que ça voulait dire que de sous-estimer un Malfoy.

Drago s'était éveillé à la science en une seule journée. Harry avait dit que ce n'était pas censé arriver avant des mois.

Mais bien sûr, quand on était Malfoy, on se destinait à être un scientifique plus puissant que tous les non-Malfoy.

Drago apprendrait donc toutes les méthodes de la rationalité de Harry, et quand le moment viendrait…
%  LocalWords:  raco ya’d Dumbledoring Lawliet Luit Yeroen Nikkie Yeroen’s
%  LocalWords:  y’know
